%% Generated by Sphinx.
\def\sphinxdocclass{report}
\documentclass[a4paper,10pt,oneside,french]{sphinxmanual}
\ifdefined\pdfpxdimen
   \let\sphinxpxdimen\pdfpxdimen\else\newdimen\sphinxpxdimen
\fi \sphinxpxdimen=.75bp\relax

\PassOptionsToPackage{warn}{textcomp}
\usepackage[utf8]{inputenc}
\ifdefined\DeclareUnicodeCharacter
% support both utf8 and utf8x syntaxes
  \ifdefined\DeclareUnicodeCharacterAsOptional
    \def\sphinxDUC#1{\DeclareUnicodeCharacter{"#1}}
  \else
    \let\sphinxDUC\DeclareUnicodeCharacter
  \fi
  \sphinxDUC{00A0}{\nobreakspace}
  \sphinxDUC{2500}{\sphinxunichar{2500}}
  \sphinxDUC{2502}{\sphinxunichar{2502}}
  \sphinxDUC{2514}{\sphinxunichar{2514}}
  \sphinxDUC{251C}{\sphinxunichar{251C}}
  \sphinxDUC{2572}{\textbackslash}
\fi
\usepackage{cmap}
\usepackage[T1]{fontenc}
\usepackage{amsmath,amssymb,amstext}
\usepackage{babel}



\usepackage{times}
\expandafter\ifx\csname T@LGR\endcsname\relax
\else
% LGR was declared as font encoding
  \substitutefont{LGR}{\rmdefault}{cmr}
  \substitutefont{LGR}{\sfdefault}{cmss}
  \substitutefont{LGR}{\ttdefault}{cmtt}
\fi
\expandafter\ifx\csname T@X2\endcsname\relax
  \expandafter\ifx\csname T@T2A\endcsname\relax
  \else
  % T2A was declared as font encoding
    \substitutefont{T2A}{\rmdefault}{cmr}
    \substitutefont{T2A}{\sfdefault}{cmss}
    \substitutefont{T2A}{\ttdefault}{cmtt}
  \fi
\else
% X2 was declared as font encoding
  \substitutefont{X2}{\rmdefault}{cmr}
  \substitutefont{X2}{\sfdefault}{cmss}
  \substitutefont{X2}{\ttdefault}{cmtt}
\fi


\usepackage[Sonny]{fncychap}
\ChNameVar{\Large\normalfont\sffamily}
\ChTitleVar{\Large\normalfont\sffamily}
\usepackage{sphinx}

\fvset{fontsize=\small}
\usepackage{geometry}


% Include hyperref last.
\usepackage{hyperref}
% Fix anchor placement for figures with captions.
\usepackage{hypcap}% it must be loaded after hyperref.
% Set up styles of URL: it should be placed after hyperref.
\urlstyle{same}

\usepackage{sphinxmessages}
\setcounter{tocdepth}{3}
\setcounter{secnumdepth}{3}


\title{Diacamma Pro}
\date{janv. 30, 2020}
\release{2.4.2}
\author{sd-libre}
\newcommand{\sphinxlogo}{\sphinxincludegraphics{DiacammaPro.jpg}\par}
\renewcommand{\releasename}{Version}
\makeindex
\begin{document}

\ifdefined\shorthandoff
  \ifnum\catcode`\=\string=\active\shorthandoff{=}\fi
  \ifnum\catcode`\"=\active\shorthandoff{"}\fi
\fi

\pagestyle{empty}
\sphinxmaketitle
\pagestyle{plain}
\sphinxtableofcontents
\pagestyle{normal}
\phantomsection\label{\detokenize{index::doc}}



\chapter{Diacamma Pro}
\label{\detokenize{pro/index:diacamma-pro}}\label{\detokenize{pro/index::doc}}
Présentation du logiciel Diacamma Pro.


\section{Présentation}
\label{\detokenize{pro/presentation:presentation}}\label{\detokenize{pro/presentation::doc}}

\subsection{Description}
\label{\detokenize{pro/presentation:description}}
\sphinxstyleemphasis{Diacamma Pro} est un logiciel de gestion spécialement conçu pour les TPE et les micro\sphinxhyphen{}entreprise.

L’application de base est entièrement gratuite et vous permet de gérer les accès à vos données et au carnet d’adresses de votre association. Les modules complémentaires vous permettront d’adapter en quelques clics le logiciel à vos besoins.

Les différents modules disponibles vous permettront, par exemple, de:
\begin{itemize}
\item {} 
Gérer vos documents de façon centralisée grâce à la gestion documentaire.

\item {} 
Gérer la comptabilité de votre association.

\item {} 
Gérer vos devis, factures et factures proformat pour les adhérents subventionnés par un CE, entre autre.

\end{itemize}

Ce manuel vous aidera dans l’utilisation de ce logiciel.
Si malgré tout, vous ne trouvez pas la réponse à vos problèmes, visiter notre site \sphinxurl{https://www.diacamma.org} où vous trouverez des tutoriels et des astuces.


\subsection{Installation}
\label{\detokenize{pro/presentation:installation}}
Vous pouvez installer \sphinxstyleemphasis{Diacamma Pro} sur un ordinateur dédié à votre association que ce soit un Apple Macintosh (OS X 10.8 et +) ou bien un PC sous MS\sphinxhyphen{}Windows (7 et +) ou sous GNU Linux (Ubuntu 14.04 ou +).

\sphinxstyleemphasis{Diacamma Pro} est un logiciel client/serveur : vous pouvez l’installer sur un ordinateur centralisateur et accéder aux données depuis un autre PC connecté au premier, sans limite du nombre d’utilisateurs simultanés.
Si le PC contenant les données est connecté de manière permanente à internet, vous aurez accès à vos données depuis n’importe où dans le monde!

Cette organisation est particulièrement intéressante pour permette à plusieurs cadres associatifs d’avoir accès à des données communes.

Quel responsable ne s’est pas arraché les cheveux suite à un échanges de documents via une clef USB où certaines modifications importantes se perdent?

Pour plus d’information, visiter notre site \sphinxurl{https://www.diacamma.org}


\subsection{Aides et support}
\label{\detokenize{pro/presentation:aides-et-support}}
Sur le site officiel du logiciel, \sphinxurl{https://www.diacamma.org}, où vous trouverez des tutoriels et un forum pour échanger des astuces entre utilisateurs.

Si vous souhaitez un service et un support plus personnalisé, vous pouvez faire confiance à notre partenaire officiel SLETO.
Pour en savoir plus sur des solutions d’hébergement et de support, rendez vous sur \sphinxurl{https://www.sleto.net}


\section{Prise en main}
\label{\detokenize{pro/first_step:prise-en-main}}\label{\detokenize{pro/first_step::doc}}
Le logiciel \sphinxstyleemphasis{Diacamma Pro} comprends un grand nombre de paramétrages et peu paraître difficile à configurer au besoin de votre structure.

Nous vous proposons cette explication pour vous aider à franchir cette première étape dans l’utilisation de cet outil.

Suivez pas à pas les différents phases de réglages. Dans chaque étape, nous ne ré\sphinxhyphen{}détaillons pas les fonctionnalités.
Nous vous invitons également à vous référer au reste du manuel utilisateur pour cela.

Il peut être intéressant de réaliser des sauvegardes au cours de cette procédure.
Cela vous permettra, si vous faite une erreur, de revenir à une étape précédente sans tout recommencer (installation comprise).


\subsection{Vérifiez la mise à jours de votre logiciel}
\label{\detokenize{pro/first_step:verifiez-la-mise-a-jours-de-votre-logiciel}}
Commencez par vérifiez que votre logiciel est à jours.
En effet, nous diffusons régulièrement des correctifs qui ne sont pas toujours inclus dans les installateurs.


\subsection{Présentation de votre structures}
\label{\detokenize{pro/first_step:presentation-de-votre-structures}}\begin{quote}

Menu \sphinxstyleemphasis{Général/Nos coordonnées}
\end{quote}

Dans cette écran, vous pouvez décrire les coordonnées de votre structure.
De nombreuses fonctionnalités utilisent ces informations en particulier pour les impressions (facture, comptabilité, listing d’adhérents…)


\subsection{Réglage de votre facturier}
\label{\detokenize{pro/first_step:reglage-de-votre-facturier}}\begin{quote}
\begin{quote}

Menu \sphinxstyleemphasis{Administration/Modules (conf.)/Configuration du règlement}

Menu \sphinxstyleemphasis{Administration/Modules (conf.)/Configuration commercial du facturier}
\end{quote}

Menu \sphinxstyleemphasis{Administration/Modules (conf.)/Configuration financière du facturier}
\end{quote}

Vérifiez que les paramètres du facturier vous correspondent.
Vous pouvez aussi ici, configurer le RIB de votre compte bancaire principal ainsi qu’en ajouter des secondaires.
Ces derniers vous seront utiles pour la fonctionnalité « dépôt de chèques » du facturier.


\subsection{Création de votre premier exercice comptable}
\label{\detokenize{pro/first_step:creation-de-votre-premier-exercice-comptable}}\begin{quote}

Menu \sphinxstyleemphasis{Administration/Modules (conf.)/Configuration comptables}

Menu \sphinxstyleemphasis{Finance/Comptabilité/plan comptable}
\end{quote}

Ouvrer votre premier exercice comptable et rendez le actifs.
Vous devrez aussi créer le plan comptable de cette exercice pour avoir une comptabilité pleinement opérationnel.


\subsection{Mise à jours comptable}
\label{\detokenize{pro/first_step:mise-a-jours-comptable}}\begin{quote}

Menu \sphinxstyleemphasis{Finance/Comptabilité/écritures comptable}

Menu \sphinxstyleemphasis{Finance/Comptabilité/Modèle d’écriture}
\end{quote}

Si vous mettez en place \sphinxstyleemphasis{Diacamma Asso} au cours de votre exercice, vous devrez également saisir votre report à nouveau de l’exercice précédent ainsi que ressaisir les écritures du début d’année.
Attention: n’oubliez pas que l’ajout d’une cotisation d’adhérent génère une facture ainsi que des écritures comptables associées. Prenez en compte dans la reprise de votre comptabilité.
Pour vous aidez dans la saisie de votre comptabilité, nous vous conseillons d’utiliser les modèles d’écritures. Enregistrez entant que modèle les écritures récurrentes que vous avez au cours d’une année. Ainsi vous pouvez rapidement compléter votre comptabilité en quelques cliques.


\subsection{Le courriel}
\label{\detokenize{pro/first_step:le-courriel}}\begin{quote}

Menu \sphinxstyleemphasis{Administration/Modules (conf.)/Paramètrages de courriel}
\end{quote}

Définissez vos réglages pour votre courriel.
Le serveur smpt permettra à \sphinxstyleemphasis{Diacamma Asso} d’envoyé un certain nombre de message: facture en PDF, mot de passe de connexion, …
Vous pouvez préciser comment réagis les liens “écrire à tous” réagis avec votre logiciel de messagerie.


\subsection{La gestions documentaire}
\label{\detokenize{pro/first_step:la-gestions-documentaire}}\begin{quote}

Menu \sphinxstyleemphasis{Administration/Modules (conf.)/Dossier}

Menu \sphinxstyleemphasis{Bureautique/Documents/Documents}
\end{quote}

Définissez vos différents dossier vous permettant d’importer vos documents à classer et à partager.

Une fois avoir parcouru ces points, votre logiciel \sphinxstyleemphasis{Diacamma Pro} est pleinement opérationnel.
N’hésitez pas à consulter le forum: de nombreuses astuces peux vous aider pour utiliser au mieux votre logiciel.


\chapter{Diacamma comptabilité}
\label{\detokenize{accounting/index:diacamma-comptabilite}}\label{\detokenize{accounting/index::doc}}
Aide relative aux fonctionnalités comptables.


\section{Definitions}
\label{\detokenize{accounting/definition:definitions}}\label{\detokenize{accounting/definition::doc}}\begin{quote}

\sphinxstylestrong{Remarques:} Ce module comptable est proche d’une comptabilité type « entreprise », néanmoins elle ne respecte pas certaines exigences légales et fiscale en la matiére.
Ce modules ne peux pas étre utilisé pour la tenu de compte de structures commerciales, concurrentielles ou professionnelles mais seulement des structures de type associative gérées par des bénévoles.
Le représentant légale de la structure utilisant ce module doit s’assurer que sa comptabilité respecte alors la législation de son pays en vigueur.
\end{quote}


\subsection{Exercice comptable}
\label{\detokenize{accounting/definition:exercice-comptable}}
Un exercice comptable est une période de temps sur laquelle une
personne morale (entreprise, association ou autre) enregistre tous les
mouvements d’argent la concernant.

Cette période est généralement de 12 mois consécutifs du 1er janvier au 31 décembre mais peut varier
d’une entité à une autre. La durée légale est toutefois fixée à un
maximum de 2 ans. La durée de l’exercice est fixée à l’avance et ne
peut être modifiée que sur décision du conseil d’administration.


\subsection{Tiers comptable}
\label{\detokenize{accounting/definition:tiers-comptable}}
Un tiers comptable est une personne physique ou morale avec
laquelle une entité va avoir des échanges monétaires (clients,
fournisseurs, salariés, administrations…).


\subsection{Journaux comptables}
\label{\detokenize{accounting/definition:journaux-comptables}}
Un journal comptable est un regroupement d’écritures comptables permettant de classer celles\sphinxhyphen{}ci.

Les journaux par défaut sont:
\begin{itemize}
\item {} 
journal d’achat contenant toutes les écritures relatives aux achats fait par une entité

\item {} 
journal de vente contenant toutes les écritures relatives aux dépenses effectuées par une entité

\item {} 
journal des encaissements contenant toutes les écritures relatives aux mouvement sur les comptes en monétaire (compte bancaires, compte caisse…) en relation avec les dépenses et recettes de l’entité

\item {} 
journal des reports à nouveau contenant les écritures permettant le passage d’un exercice comptable à son suivant

\item {} 
journal des opérations diverses contenant l’ensemble des autres écritures (ex: frais financiers…)

\end{itemize}


\subsection{Ecritures comptables}
\label{\detokenize{accounting/definition:ecritures-comptables}}
Une écriture comptable est un ensemble de lignes inscrites dans divers
comptes comptables permettant un équilibre.
La somme des crédits d’une écriture doit donc être égale à la somme des débits de cette même écriture.

Par exemple, une écriture d’achat se schématise par:
\begin{itemize}
\item {} 
une ligne au crédit du compte tiers fournisseur représentant l’ensemble de la somme de la facture

\item {} 
une ou plusieurs lignes au débit des comptes de charges correspondants au type de ressources achetées (matériel, service…)

\end{itemize}

Le total des lignes dans les comptes de charge est donc égal au montant
porté sur la ligne de compte tiers fournisseur.

Les écritures comptables d’encaissement peuvent et doivent être pointées afin de marquer le rapprochement avec les comptes bancaires et
la caisse physique. De cette façon, on peut suivre facilement les écritures passées dans la comptabilité mais non encore effectives dans
la réalité. Le pointage est nécessaire pour le passage d’un exercice comptable à son suivant.

Il est également possible et recommandé de lettrer les écritures, c’est à dire de créer un sous ensemble
cohérent d’écritures en provenance de journaux divers afin de stipuler qu’elle correspondent à la même opération de la vie réelle.

Ex: l’écriture comptable d’achat d’un bien peut être lettrée avec son écriture d’encaissement.


\subsection{Plan comptable de l’exercice}
\label{\detokenize{accounting/definition:plan-comptable-de-l-exercice}}
Le plan comptable de l’exercice est l’ensemble des comptes utilisés au
cours d’un exercice comptable en se basant sur le plan comptable
couramment admis par l’administration fiscale.

les numéros de comptes doivent impérativement commencer par le préfixe donné par le
plan comptable en vigueur au moment de la création du compte.


\section{Exercice comptable}
\label{\detokenize{accounting/fiscalyear:exercice-comptable}}\label{\detokenize{accounting/fiscalyear::doc}}

\subsection{Paramétrages}
\label{\detokenize{accounting/fiscalyear:parametrages}}\begin{quote}

Menu \sphinxstyleemphasis{Administration/Modules (conf.)/Configuration comptable}
\end{quote}

Ouvrez l’onglet « Paramètres » et éditez\sphinxhyphen{}les avec le bouton « Modifier ». Paramétrez la devise et sa précision, la taille des codes comptables. Précisez aussi si vous avez l’intention ou non de mettre en place une comptabilité analytique.
\begin{quote}

\noindent\sphinxincludegraphics{{parameters}.png}
\end{quote}


\subsection{Création d’un exercice comptable}
\label{\detokenize{accounting/fiscalyear:creation-d-un-exercice-comptable}}
Lors de la mise en place de votre comptabilité sous \sphinxstyleemphasis{Diacamma Syndic}, vous aurez à spécifier le système comptable qui sera utilisé (ex. Plan comptable général français). \sphinxstylestrong{Attention :} une fois choisie, cette option ne sera plus modifiable.
\begin{quote}
\begin{quote}

Menu \sphinxstyleemphasis{Administration/Modules (conf.)/Configuration comptable} \sphinxhyphen{} Onglet « Liste d’exercices »
\end{quote}

\noindent\sphinxincludegraphics{{fiscalyear_list}.png}
\end{quote}

Vérifiez que le système comptable a été choisi et cliquez sur « + Créer » afin de renseigner les bornes du nouvel exercice.
\begin{quote}

\noindent\sphinxincludegraphics{{fiscalyear_create}.png}
\end{quote}

Pour le premier exercice sous \sphinxstyleemphasis{Diacamma Syndic}, saisissez la date de début et la date de fin de l’exercice puis cliquez sur le bouton « OK ». Les exercices suivants ont comme date de début le lendemain de la date de clôture de l’exercice précédent et seule la date de fin est à saisir.

Notez que le logiciel associe à chaque exercice un répertoire de stockage du gestionnaire de documents : certains documents
officiels seront sauvegardés dans celui\sphinxhyphen{}ci. Le bouton « Contrôle » vous permet à tout moment de  vérifier que vos documents officiels ont bien été générés.

Votre nouvel exercice figure maintenant dans la liste des exercices. Il est \sphinxstylestrong{{[}en création{]}}. Lorsque plusieurs exercices ont été créés, vous devez activer celui sur lequel vous souhaitez travailler par défaut, à l’aide du bouton « Activé ».
\begin{quote}

\sphinxstyleemphasis{Onglet « Journaux » et Onglet « Champ personnalisé des tiers »}
\end{quote}

Depuis ce même écran de configuration, vous pouvez également modifier ou ajouter des journaux. Des champs personnalisés peuvent aussi être ajoutés à la fiche modèle de tiers comptable. Ceci peut être intéressant si vous voulez réaliser des recherches/filtrages sur des informations propres à votre fonctionnement.

Maintenant, vous devez fermer la fenêtre « Configuration comptable » et créer le plan comptable de votre structure.
\begin{quote}
\begin{quote}

Menu \sphinxstyleemphasis{Comptabilité/Gestion comptable/Plan comptable}
\end{quote}
\end{quote}

Avec le bouton « + Initiaux », générez automatiquement votre propre plan comptable général à partir du plan de comptes type fourni par le logiciel.
Adaptez celui\sphinxhyphen{}ci aux besoins de votre structure avec les boutons « Ajouter » et « Supprimer ».
\begin{itemize}
\item {} \begin{description}
\item[{\sphinxstylestrong{Première tenue de comptabilité sous Diacamma Syndic}}] \leavevmode{[}vous migrez sous Diacamma et avez des à\sphinxhyphen{}nouveaux à saisir.{]}
Avec le bouton « + Initiaux », générez automatiquement votre propre plan comptable général à partir du plan de comptes type.
Adaptez celui\sphinxhyphen{}ci aux besoins de votre structure avec les boutons « Ajouter » et « Supprimer ».
Quittez l’écran \sphinxstyleemphasis{Plan comptable} et ouvrez le menu \sphinxstyleemphasis{Comptabilité/Gestion comptable/Ecritures comptables}.
Saisissez vos soldes à\sphinxhyphen{}nouveaux en une seule écriture en prenant bien soin de la contrôler dans le journal « Report à nouveau ».
Ceci fait, réouvrez le menu \sphinxstyleemphasis{Comptabilité/Gestion comptable/Plan comptable}.

\end{description}

\item {} \begin{description}
\item[{\sphinxstylestrong{Exercice comptable suivant}}] \leavevmode{[}Il ne s’agit pas de votre premier exercice sous \sphinxstyleemphasis{Diacamma}.{]}
Si ce n’est pas déjà réalisé, avec le bouton « Importer » (et non « Initiaux »), vous devez importer le plan comptable de l’exercice précédent.
Contrôlez l’importation et mettez à jour, si besoin, le plan comptable de l’exercice.
Suite à votre dernière assemblée générale, les excédents n\sphinxhyphen{}1 doivent être normalement ventilés avant clôture de l’exercice n\sphinxhyphen{}1. Si ce n’est pas le cas, vous devez passer cette écriture.
Clôturez l’exercice n\sphinxhyphen{}1. Son état est maintenant \sphinxstylestrong{{[}terminé{]}}. Le nouvel exercice est toujours \sphinxstylestrong{{[}en création{]}}.
Utilisez le bouton « Report à nouveau » afin que les soldes n\sphinxhyphen{}1 des comptes d’actif et de passif soient repris dans la comptabilité du nouvel exercice. Vous pouvez constater à l’écran que les soldes des comptes de bilan non soldés fin n\sphinxhyphen{}1 ont été repris. L’écriture correspondante  au journal « Report à nouveau » est générée et validée automatiquement.

\end{description}

\end{itemize}

\sphinxstylestrong{Afin d’achever l’ouverture de votre nouvel exercice, vous devez maintenant cliquez sur « Commencer ».}
\sphinxstylestrong{Votre exercice est maintenant {[}en cours{]}}.


\subsection{Création, modification et édition de comptes du plan comptable}
\label{\detokenize{accounting/fiscalyear:creation-modification-et-edition-de-comptes-du-plan-comptable}}\begin{quote}

Menu \sphinxstyleemphasis{Comptabilité/Gestion comptable/Plan comptable}
\end{quote}

A tout moment vous pouvez ajouter un nouveau compte dans votre plan comptable.
\begin{quote}

\noindent\sphinxincludegraphics{{account_new}.png}
\end{quote}

Référez\sphinxhyphen{}vous aux codes définis par la règlementation de votre pays. Comme déjà écrit, pour la France, votre plan comptable doit respecter la nomenclature des comptes énoncée dans l’arrêté du 14 mars 2005 relatif aux comptes du syndicat des copropriétaires. Des subdivisions sont possibles.

Un compte peut être modifié, tant pour ce qui est de son numéro que de son intitulé. Les imputations (lignes d’écritures) qui lui sont associées seront automatiquement modifiées. Le changement n’est permis que si le nouveau compte relève de la même catégorie comptable (charge, produit…).

Lorsque vous consultez un compte (bouton « Editer » ou double\sphinxhyphen{}clic), les écritures associées au compte sont affichées.
\begin{quote}

\noindent\sphinxincludegraphics{{account_edit}.png}
\end{quote}

Il vous est aussi possible de supprimer un compte du plan comptable à la condition qu’aucune écriture ne lui soit associée.


\subsection{Clôture d’un exercice}
\label{\detokenize{accounting/fiscalyear:cloture-d-un-exercice}}
En fin d’exercice comptable, celui\sphinxhyphen{}ci est clôturé. Cette opération, définitive, se réalise sous le contrôle de votre
vérificateur aux comptes.

Au préalable, vous devez :
\begin{itemize}
\item {} 
Passer vos écritures d’inventaire (charges à payer, produits à recevoir, créances douteuses…)

\item {} 
Contrôler que toutes les charges et les produits ont bien été imputés en comptabilité analytique

\item {} 
Vérifier que vos dépenses et vos recettes sont bien ventilées sur vos différentes catégories

\item {} 
Vérifier que toutes vos dépenses ont été ventilées sur les copropriétaires

\item {} 
Affecter vos excédents conformément aux délibérations de votre assemblée générale

\item {} 
Valider les écritures provisoires au brouillard

\item {} 
Lettrer les comptes de tiers

\item {} 
Créer l’exercice suivant si cela n’a pas été réalisé

\item {} 
Sauvegarder votre dossier
\begin{quote}

Menu \sphinxstyleemphasis{Comptabilité/Gestion comptable/Plan comptable}
\end{quote}

\end{itemize}

Cliquez sur le bouton « Clôturer ».

La clôture a pour effet de :
\begin{itemize}
\item {} 
Solder les comptes de gestion

\item {} 
Interdire tout ajout d’écriture

\item {} 
Arrêter les comptes de bilan et les comptes de tiers (copropriétaires, fournisseurs…)

\item {} 
Assurer qu’il ne pourra plus être apporté de modification à l’exercice clôturé

\end{itemize}

\sphinxstylestrong{Remarques :}
\begin{itemize}
\item {} 
Tant qu’un exercice n’est pas clôturé, vous pouvez enregistrer des opérations sur celui\sphinxhyphen{}ci et le suivant

\item {} 
Certaines structures ont des règles de clôture spécifique (exemple les ASL): bien verifier votre règlementation comptable en la matière.

\end{itemize}


\section{Tiers comptable}
\label{\detokenize{accounting/third:tiers-comptable}}\label{\detokenize{accounting/third::doc}}

\subsection{Création d’un tiers}
\label{\detokenize{accounting/third:creation-d-un-tiers}}
Plaçons nous dans le menu \sphinxstyleemphasis{Comptabilité/Tiers}.

\noindent\sphinxincludegraphics{{third_list}.png}

La liste des tiers précédemment enregistrés apparaît.
Vous pouvez réaliser un certain nombre de fitrage rapide suivant le nom ou
la situation. Vous pouvez alors imprimer la liste.
Pour ajouter un nouveau tiers, vous devez commencer par choisir un contact (physique
ou moral) associé à ce tiers comptable.

\noindent\sphinxincludegraphics{{third_add}.png}

Depuis cet écran, vous pouvez aussi créer un nouveau contact avant de le sélectionner.

\noindent\sphinxincludegraphics{{third_edit}.png}

Pour chaque tiers, vous pouvez associer des comptes comptables
correspondant à la nature de vos tiers: fournisseur, client, salarié et
sociétaire. Vous pouvez changer ces comptes pour imputer dans votre
comptabilité comme vous le souhaitez cette personne au cours
d’opérations financières.


\subsection{Situation d’un tiers}
\label{\detokenize{accounting/third:situation-d-un-tiers}}
La fiche d’un tiers vous permet d’avoir une vue globale de l’état des recettes et dépenses liées à ce tiers.

\noindent\sphinxincludegraphics{{third_state}.png}

Vous retrouverez ici l’ensemble des écritures comptables de
l’exercice liées à ce tiers. Vous trouverez également un résumé des
débits et crédits permettant en un seul regard de savoir s’il reste des
dettes impayées. Avec d’autres modules financiers, vous pourrez
également consulter des opérations liées.


\subsection{Configuration}
\label{\detokenize{accounting/third:configuration}}
Depuis le menu \sphinxstyleemphasis{Administration/Modules (conf.)/Configuration comptable} vous avez la possibilité d’ajouter à tout tiers des champs personnalisés.
Le mécanisme est similaire à ce que vous pouvez trouver dans la configuration des contacts.


\section{Écritures}
\label{\detokenize{accounting/entity:ecritures}}\label{\detokenize{accounting/entity::doc}}
Avant toute saisie d’écriture, assurez\sphinxhyphen{}vous de l’existence de vos journaux :
\begin{quote}

Menu \sphinxstyleemphasis{Administration/Modules(conf.)/Configuration comptable} onglet « Journaux »
\end{quote}

De même, vous devez contrôler que votre plan comptable est à jour.
\begin{quote}

Menu \sphinxstyleemphasis{Comptabilité/Gestion comptable/Plan comptable}
\end{quote}


\subsection{Saisie d’une écriture}
\label{\detokenize{accounting/entity:saisie-d-une-ecriture}}

\subsubsection{Cas général}
\label{\detokenize{accounting/entity:cas-general}}\begin{quote}
\begin{quote}

Menu \sphinxstyleemphasis{Comptabilité/Gestion comptable/Écritures comptables}
\end{quote}

\noindent\sphinxincludegraphics{{entity_list}.png}
\end{quote}

Depuis cet écran, vous avez la possibilité de visualiser les écritures précédemment saisies et vous pouvez en ajouter de nouvelles.
A l’écran, vous pouvez aussi consulter les écritures saisies après les avoir filtrées sur l’exercice comptable, un journal (ou tous)  et/ou sur l’état des écritures :
\begin{itemize}
\item {} 
Tout : aucun filtrage n’est appliqué

\item {} 
En cours (brouillard): seulement les écritures provisoires (non encore validées)

\item {} 
Validée : seulement les écritures déjà validées

\item {} 
Lettrée : seulement les écritures lettrées

\item {} 
Non lettrée : seulement les écritures non encore lettrées

\end{itemize}

Pour saisir une nouvelle écriture, cliquez sur le bouton « + Ajouter ».
Sélectionnez le journal et saisissez la date de l’opération ainsi que le libellé de l’écriture (pièce et numéro…). Cliquez sur le bouton « Modifier ».

Pour chaque ligne de l’écriture :
\begin{itemize}
\item {} 
Saisissez les premiers chiffres du numéro du compte devant être mouvementé et sélectionnez\sphinxhyphen{}le dans votre plan comptable. Un compte doit exister dans le plan comptable de l’exercice pour être mouvementable

\item {} 
Si demandé, sélectionnez le tiers et le code analytique de rattachement

\item {} 
Saisissez le montant au débit ou au crédit du compte spécifié

\item {} 
Cliquez sur le bouton « + Ajouter ».

\end{itemize}

Ceci fait, cliquez sur « Ok ». Vous ne pourrez valider la saisie de votre écriture que si elle est équilibrée (débits = crédits).

\sphinxstylestrong{Après clôture de l’exercice comptable, il ne sera plus possible de passer une écriture sur l’exercice clos. Avant cela, prenez soin de passer vos dernières écritures.}

Une écriture étant provisoire, le bouton « Inverser » vous permet d’inverser très facilement votre écriture si besoin.
Quand on débute en comptabilité, on a parfois du mal à savoir si un compte doit être débité ou crédité.
En saisie, lorsque vous débitez un compte fournisseur et créditez un compte de charge, un message vous alerte sur le fait que vous avez probablement saisi l »écriture d’un avoir ».


\subsubsection{Comptabiliser un règlement}
\label{\detokenize{accounting/entity:comptabiliser-un-reglement}}
Un règlement peut être saisi manuellement comme précédemment. Mais bien souvent il est lié à une opération déjà comptabilisée  (ex. achat, appel de fonds) et constatant une dette ou une créance.

Pour simplifier votre saisie, éditez l’écriture constatant la dette ou la créance réglée. Cliquez sur le bouton « Règlement » : l’application vous propose alors une nouvelle écriture partiellement remplie avec le compte du tiers débité (règlement de dette) ou crédité (règlement de créance).
Il ne vous reste plus qu’à préciser sur quel compte financier (caisse, banque…) vous voulez imputer le règlement et à contrôler l’écriture générée après l’avoir complétée.

Une fois l’écriture de règlement validée via cette fonctionnalité, l’écriture d’origine de la dette ou de la créance et celle du règlement sont automatiquement associées. Les lignes concernant le compte de tiers seront lettrées automatiquement.


\subsubsection{Écriture d’à\sphinxhyphen{}nouveaux}
\label{\detokenize{accounting/entity:ecriture-d-a-nouveaux}}
Après clôture d’un exercice comptable, l’écriture d’à\sphinxhyphen{}nouveaux est automatiquement générée lors de la phase d’initialisation de l’exercice suivant. Cette écriture, passée dans le journal « Report à nouveau », est automatiquement validée.
A ce moment, vous pouvez être amené à saisir des opérations spécifiques comme par exemple la ventilation des excédents de l’année précédente.

Par contre, dans ce journal, vous ne pouvez pas enregistrer de charges ou de produits.


\subsection{Lettrage d’écritures}
\label{\detokenize{accounting/entity:lettrage-d-ecritures}}
Comme nous l’avons évoqué dans un précédent chapitre, il est fréquent que des mouvements enregistrés en comptabilité trouvent leur source dans une ou plusieurs opérations liées. Dans ce cas, les lignes d’écritures correspondantes peuvent être lettrées.

Toutefois, trois conditions pour que des lignes d’écriture soient lettrables :
\begin{itemize}
\item {} 
Elles doivent concerner le même compte de tiers (fournisseur ou copropriétaire)

\item {} 
Pour être lettrables, le total des débits doit être égal au total des crédits

\item {} 
Seules les lignes d’un même exercice sont lettrables. C’est pourquoi il faut lettrer les comptes de tiers avant clôture de l’exercice

\end{itemize}

Les lignes lettrées conjointement se voient attribuer le même code lettre.

Toute ligne lettrée peut être délettrée.


\subsection{Validation d’écritures}
\label{\detokenize{accounting/entity:validation-d-ecritures}}
Par défaut, une écriture est saisie au brouillard, ce qui permet de la modifier ou de la supprimer tant qu’elle n’est pas validée.
Cette écriture doit être validée pour entériner votre saisie. En principe, cette validation est confiée à la personne en charge de la vérification de la comptabilisation des opérations.

Pour réaliser cette action, sélectionnez les écritures contrôlées et cliquez sur le bouton « Clôturer »: L’application affectera alors un numéro aux écritures validées ainsi qu’une date de validation.

Une fois validée, une écriture devient non modifiable : ce mécanisme assure le caractére intangible et irréversible de votre comptabilité.

Comme l’erreur est humaine, l’écriture validée ne pouvant pas être modifiée ou supprimée, vous devrez procéder comme suit :
\begin{itemize}
\item {} 
1 : Contrepasser l’écriture erronée en créant une écriture inverse pour l’annuler. Le libellé doit spécifier la référence de l’écriture annulée avec la mention « Contrepassation… »

\item {} 
2 : Enregistrer l’écriture correcte

\end{itemize}

\sphinxstylestrong{Avant clôture de l’exercice comptable, toutes les écritures doivent étre validées}.


\subsection{Recherche d’écriture(s)}
\label{\detokenize{accounting/entity:recherche-d-ecriture-s}}
Depuis la liste des écritures, le bouton « Recherche » vous permet de définir les critères de recherche d’écritures comptables.
\begin{quote}

\noindent\sphinxincludegraphics{{entity_search}.png}
\end{quote}

En cliquant sur « Recherche », l’outil va rechercher dans la base toutes les écritures satisfaisant aux critères saisis.
La ou les écritures extraites pourront être :
\begin{itemize}
\item {} 
Imprimées

\item {} 
Éditées/modifiées

\item {} 
Clôturée, lettrées ou délettrées…

\end{itemize}


\subsection{Import d’écritures}
\label{\detokenize{accounting/entity:import-d-ecritures}}
Depuis la liste des écritures, le bouton « Import » vous permet d’importer des écritures comptables depuis un fichier CSV.

Après avoir sélectionné l’exercice d’import, le journal et les informations de format de votre fichier CSV, vous devez associer les champs des écritures aux colonnes de votre document (la première ligne de votre document doit décrire la nature de chaque colonne).
\begin{quote}

\noindent\sphinxincludegraphics{{entity_import}.png}
\end{quote}

Vous pouvez alors contrôler vos données avant de les valider.
L’import réalisé, l’outil vous présentera le résultat des écritures réellement importées.

\sphinxstylestrong{Notez que les lignes d’écritures ne seront pas importées si :}
\begin{itemize}
\item {} 
Le code comptable précisé n’existe pas dans le plan comptable de l’exercice

\item {} 
La date n’est pas inclue dans l’exercice comptable actif

\item {} 
Le principe de la partie double n’est pas respecté car pour toute opération, le total des débits doit être égal au total des crédits

\end{itemize}

Bien que cela ne bloque pas l’import, le tiers et le code analytique seront laissés vides si ceux indiqués ne sont pas référencés dans votre dossier comptable. Vous devez donc contrôler l’importation et la modifier si besoin.


\section{Comptabilité analytique}
\label{\detokenize{accounting/costaccounting:comptabilite-analytique}}\label{\detokenize{accounting/costaccounting::doc}}
Pour réaliser une analyse financière des différentes activités de votre structure et déterminer le résultat de chacune d’elles, vous pouvez mettre en place une comptabilité analytique.
\begin{quote}

Menu \sphinxstyleemphasis{Administration/Modules(conf.)/Configuration comptable}
\end{quote}

Après avoir ouvert l’onglet « Paramètres » \sphinxhyphen{} bouton « Modifier », cochez le paramètre \sphinxstyleemphasis{Comptabilité analytique}. A noter qu’en activant ce paramètre, toutes les charges et les produits devront obligatoirement être repris en comptabilité analytique.

La comptabilité analytique proposée par le logiciel est une version simplifiée. En effet, il n’est pas possible de ventiler une charge ou un produit sur plusieurs codes analytiques.


\subsection{Les codes analytiques}
\label{\detokenize{accounting/costaccounting:les-codes-analytiques}}\begin{quote}

Menu \sphinxstyleemphasis{Comptabilité/Gestion comptable/Comptabilités analytiques}
\end{quote}

Vous accédez à la liste des codes analytiques créés.

Depuis cet écran vous pouvez créer, modifier ou supprimer un code analytique.

Chaque code a un titre, un descriptif et un statut (ouvert ou clôturé).
En mode « liste », figure le résultat comptable (les produits diminués des charges) de chaque code analytique.
\begin{quote}

\noindent\sphinxincludegraphics{{costaccount_list}.png}
\end{quote}

Par défaut, un filtrage vous permet de ne voir que  les codes analytiques ouverts. Vous pouvez paramétrer le filtre à l’aide des boutons de liste.


\subsection{Imputation analytique d’une charge ou d’un produit}
\label{\detokenize{accounting/costaccounting:imputation-analytique-d-une-charge-ou-d-un-produit}}\begin{quote}

Menu \sphinxstyleemphasis{Comptabilité/Gestion comptable/Ecritures comptables}
\end{quote}

Si vous avez des codes analytiques ouverts, vous pouvez imputer une charge ou un produit sur l’un d’entre eux.
\begin{quote}

\noindent\sphinxincludegraphics{{costaccount_assign}.png}
\end{quote}

Que l’écriture correspondante soit validée ou non, affichez cette écriture  et éditez\sphinxhyphen{}la.
Sélectionnez le  mouvement relatif à la charge ou au produit à imputer en analytique et bouton « Modifier ».
Renseignez le code analytique de rattachement.

Il est aussi possible de réaliser cette imputation par lot.
Sélectionnez les mouvements à affecter et cliquez sur le bouton \sphinxstyleemphasis{Analytique}. Choisissez alors le nouveau code à utiliser
pour l’ensemble des mouvements (charges ou  produits) sélectionnés.


\subsection{Impressions analytiques}
\label{\detokenize{accounting/costaccounting:impressions-analytiques}}
Depuis la liste des codes analytiques, vous pouvez réaliser un rapport type « Compte de résultat de comptabilité analytique ».
Pour cela, affichez les codes analytiques. Sélectionnez\sphinxhyphen{}les (un ou plusieurs) et cliquez sur le bouton « Rapport ».
Tout comme le Compte de résultat de la Comptabilité générale, ce rapport reprend pour chaque code analytique sélectionné, le montant des charges et des produits ainsi que le budget. Cela permet de comparer, là encore le résultat réel d’une activité au résultat attendu.


\section{Modèle}
\label{\detokenize{accounting/model:modele}}\label{\detokenize{accounting/model::doc}}

\subsection{Déclaration d’un modèle}
\label{\detokenize{accounting/model:declaration-d-un-modele}}
Certaines opérations sont régulièrement enregistrées. Pour soulager la saisie de celles\sphinxhyphen{}ci, un modèle d’écriture ou masque de saisie peut être créé et enregistré afin d’être utilisé ultérieurement en saisie.
Un modèle d’écriture peut aussi être utilisé pour faciliter la comptabilisation d’opérations complexes.

Le modèle se présente comme une écriture.
\begin{quote}
\begin{quote}

Menu \sphinxstyleemphasis{Comptabilité/Gestion comptable/Modèles d’écritures}
\end{quote}

\noindent\sphinxincludegraphics{{model_list}.png}
\end{quote}

Il est associé à un journal, contient un descriptif, précise les comptes à débiter et ceux à créditer, avec indication des montants qui ne peuvent pas être laissés nuls.
\begin{quote}

\noindent\sphinxincludegraphics{{model_item}.png}
\end{quote}

Nous vous conseillons de créer un modèle pour chacune de vos dépenses ou recettes règulières. Ainsi, vous gagnerez du temps sur la saisie de votre comptabilité et n’aurez pas à rechercher les bons codes comptables ni le sens des imputations.


\subsection{Utilisation d’un modèle}
\label{\detokenize{accounting/model:utilisation-d-un-modele}}
L’utilisation d’un modèle est très simple.

En saisie d’écriture, cliquez sur le bouton « + Modèle ».
\begin{quote}

\noindent\sphinxincludegraphics{{model_add}.png}
\end{quote}

Sélectionnez votre modèle et précisez le coefficient multiplicateur qui devra être appliqué au montant présaisi dans le modèle. Ce facteur est très pratique lorsque l’on a des factures récurrentes mais dont le montant peut fluctuer. Il est alors possible, è l’aide de ce réel, de le faire varier.
Validez votre sélection par « Ok ». Une écriture est générée d’après le modèle. Vous pouvez la corriger comme n’importe quelle écriture.


\section{Budget prévisionnel}
\label{\detokenize{accounting/budget:budget-previsionnel}}\label{\detokenize{accounting/budget::doc}}

\subsection{Budget par analytique}
\label{\detokenize{accounting/budget:budget-par-analytique}}
Depuis l’interface des comptabilités analytiques, vous pouvez ajouter un budget prévisionnel à chacun.
Cliquez simplement sur le bouton \sphinxstyleemphasis{Budget} après avoir sélectionner une comptabilité à compléter.

L’interface vous permet alors d’ajouter des comptes de charges ou de produits ainsi qu’un solde prévisionnel.
vous pouvez également importer les montants des charges et produits du résultat d’une comptabilités précédentes.

Ce budget prévisionnel apparait alors dans les rapports afin de le comparer avec la comptabilité réalisé.


\subsection{Budget par exercice}
\label{\detokenize{accounting/budget:budget-par-exercice}}
Depuis l’interface du plan comptable courant, vous pouvez ajouter un budget prévisionnel à l’exercice via le bouton \sphinxstyleemphasis{Budget}.

Comme pour le budget analytique, vous pouvez ajouter des comptes de charges ou de produits ainsi que d’importer le résultat de l’exercice précédent.
A noter qu’automatiquement, l’ensemble des budgets analytiques associés au même exercice sont automatiques consolidés dans ce budget d’exercice.

Le \sphinxstyleemphasis{resultat d’exercice} presente également la comptabilité courant en affichant également le budget prévisionnel à des fins de comparaison.


\section{Rapports}
\label{\detokenize{accounting/reporting:rapports}}\label{\detokenize{accounting/reporting::doc}}\begin{quote}

Menu \sphinxstyleemphasis{Comptabilité/Rapports comptables}
\end{quote}

Dans cette catégorie, vous accédez à l’ensemble des documents de synthèse élaborés en fin d’exercice. Les rapports sont obtenus à partir de tous les enregistrements comptables passés pendant l’exercice comptable.

En cours d’exercice, tout rapport peut être consulté à l’écran, sauvegardé dans le gestionnaire de documents au format PDF ou imprimé. Ultérieurement, vous pourrez aussi consulter un rapport sauvegardé et en réaliser une impression tel qu’il a été sauvegardé ou en le régénérant sur un autre modèle.

A la clôture de l’exercice, l’ensemble de ces rapports sont générés et sauvegardés automatiquement dans le gestionnaire de documents. Lorsque vous les éditerez, par défaut vous téléchargerez la sauvegarde. Vous pourrez régénérer un nouveau PDF sur un autre modèle. Par contre celui\sphinxhyphen{}ci comportera la mention « duplicata » en filigrane.

Dans le bilan comptable et le compte de résultat, le budget de l’exercice est reporté afin de permettre la comparaison réalisé\sphinxhyphen{}prévisionnel.


\subsection{Compte de résultat}
\label{\detokenize{accounting/reporting:compte-de-resultat}}
Le compte de résultat  synthétise l’ensemble des charges et des produits de l’exercice comptable.
Il met en évidence le résultat net, c’est\sphinxhyphen{}à\sphinxhyphen{}dire la différence entre vos produits et vos charges. Il y a excédent quand les produits excèdent les charges et inversement, un déficit.


\subsection{Bilan comptable}
\label{\detokenize{accounting/reporting:bilan-comptable}}
Le bilan comptable est une photographie (c’est un instantané) du patrimoine de l’entreprise qui permet l’évaluation d’une structure, et plus précisément de savoir après retraitement (par exemple d’une optique patrimoniale à celle sur option de juste valeur pour l’adoption des normes internationales) combien elle vaut et si elle est solvable.
Au bilan, le résultat qui y figure est égal au solde du compte de résultat (excédent ou déficit)

Il existe trois finalités au bilan comptable :
\begin{itemize}
\item {} 
Le bilan interne, généralement détaillé, utilisé par les responsables de la structure pour différentes analyses internes

\item {} 
Le bilan officiel, destiné aux contrôleurs de la comptabilité (auditeurs et commissaires aux comptes) et aux tiers (actionnaires, banques, clients, salariés, collectivités…).

\item {} 
Le bilan fiscal qui sert à déterminer le bénéfice imposable

\end{itemize}

\textless{}\textless{}\textless{}\textless{}\textless{}\textless{}\textless{} HEAD
Le bilan est une photographie du patrimoine de l’entreprise qui permet de réaliser une évaluation d’entreprise, et plus précisément de savoir après retraitement (par exemple d’une optique patrimoniale à celle sur option de juste valeur pour l’adoption des normes internationales) combien elle vaut et si elle est solvable.
Il existe donc trois finalités au bilan:
\begin{itemize}
\item {} 
Le bilan comptable interne, généralement détaillé, utilisé par les responsables de l’entreprise pour différentes analyses internes;

\item {} 
Le bilan comptable officiel, destiné aux contrôleurs de la comptabilité (auditeurs et commissaires aux comptes) et aux actionnaires (et plus généralement aux tiers);

\item {} 
Le bilan fiscal, qui sert à déterminer le bénéfice imposable;

\end{itemize}


\subsection{Grand livre}
\label{\detokenize{accounting/reporting:grand-livre}}
Le Grand livre est le recueil de l’ensemble des comptes utilisés par une structure dans le cadre de la tenue de sa comptabilité.
Il faut distinguer le grand livre général (comptes des classes 1 à 7) des grands livres auxiliaires avec le détail des comptes de tiers (clients, fournisseurs, associés, copropriétaires).

vous pouvez paramétrer vos éditions afin de les personnaliser, avec :
\begin{itemize}
\item {} 
La période (dates de début et de fin)

\item {} 
Le code comptable commençant par

\item {} 
Ecritures non\sphinxhyphen{}lettrées ou toutes

\end{itemize}

\textless{}\textless{}\textless{}\textless{}\textless{}\textless{}\textless{} HEAD
Des options de filtrage sont à votre disposition:
\begin{itemize}
\item {} 
L’exercice et la plage de dates désirées
Afin de consulter que les opérations de la période concerné.

\item {} 
Code comptable commençant par
En indiquant le début de code d’un compte, vous effectuerez un filtrage avec les opérations concernées par seulement ces comptes.

\item {} 
Seulement les écritures non\sphinxhyphen{}lettrées
En cochant cette coche, vous effectuerez n’afficherez que les opérations n’étant pas rapprocher par lettrage.
Noter que seul le lettrage de ligne d’écriture de tiers n’a de sens.

\end{itemize}


\subsection{Balance}
\label{\detokenize{accounting/reporting:balance}}
\textless{}\textless{}\textless{}\textless{}\textless{}\textless{}\textless{} HEAD
La balance comptable est un état d’une période, établi à partir de la liste de tous les comptes du grand livre de l’entreprise (qu’ils soient de bilan ou de gestion) et regroupant tous les totaux (ou masses) en débit et crédit de ces comptes et par différence tous les soldes débiteurs et créditeurs.

Des options de filtrage sont à votre disposition:
\begin{itemize}
\item {} 
L’exercice et la plage de dates désirées
Afin de consulter que les opérations de la période concerné.

\item {} 
Code comptable commençant par
En indiquant le début de code d’un compte, vous effectuerez un filtrage avec les opérations concernées par seulement ces comptes.

\item {} 
Seulement les non\sphinxhyphen{}soldés
Permet, en cochant cette coche, de n’afficher que les lignes n’ayant pas un solde nul.

\item {} 
Détail par tiers
En cochant cette coche, vous afficherez pour les comptes de tiers le détail de leur balance par tiers.

\end{itemize}

La balance générale doit être équilibrée, avec « total des débits » = « total des crédits » et « total des soldes débiteurs » = « total des soldes créditeurs ». Cet équilibre permet de vérifier que le principe de la partie double a bien été respecté lors de la comptabilisation des opérations et que tous les mouvements passés au journal ont bien été reportés dans le grand libre.

Vous pouvez personnaliser vos éditions avec :
\begin{itemize}
\item {} 
L’exercice comptable

\item {} 
La période (dates de début et de fin)

\item {} 
Le code comptable commençant par

\item {} 
Seulement les (comptes) non soldés

\end{itemize}

Vous pouvez aussi spécifier si le détail par tiers est souhaité, ce qui permet de transformer votre balance générale en balance mixte générale\sphinxhyphen{}auxiliaire.

\textgreater{}\textgreater{}\textgreater{}\textgreater{}\textgreater{}\textgreater{}\textgreater{} e82deb22517451d687808946aae3d1580350c68c


\subsection{Listing des écritures}
\label{\detokenize{accounting/reporting:listing-des-ecritures}}
Aux rapports comptables, s’ajoute l’édition des journaux :
\begin{quote}

Menu \sphinxstyleemphasis{Comptabilité/Gestion comptable/Ecritures comptables}
\end{quote}

Depuis l’écran de la liste des écritures comptables, vous avez la possibilité de les visualiser et de les exporter en PDF ou  au format CSV (ce qui permet l’import dans un tableur).


\subsection{Listing du plan comptable de l’exercice}
\label{\detokenize{accounting/reporting:listing-du-plan-comptable-de-l-exercice}}
Depuis l’écran du plan comptable de l’exercice :
\begin{quote}

Menu \sphinxstyleemphasis{Comptabilité/Gestion comptable/Plan comptable}
\end{quote}

Pour un exercice donné et par type de comptes (ou tous), vous pouvez visualiser, pour l’ensemble des comptes ouverts, le récapitulatif des soldes de début et de fin d’exercice avec l’indication des soldes de fin compte\sphinxhyphen{}tenu des seules écritures validées.

Ce récapitulatif peut être imprimé, exporté au format PDF ou CSV (ce qui permet l’import de vos soldes dans un tableur).


\chapter{Facturier Diacamma}
\label{\detokenize{invoice/index:facturier-diacamma}}\label{\detokenize{invoice/index::doc}}
Aide relative aux fonctionnalités de gestion de factures.


\section{Les articles}
\label{\detokenize{invoice/articles:les-articles}}\label{\detokenize{invoice/articles::doc}}

\subsection{Création et modification}
\label{\detokenize{invoice/articles:creation-et-modification}}
Depuis le menu \sphinxstyleemphasis{Facturier/Les articles} vous avez la possibilité de définir l’ensemble de vos articles facturables.

\noindent\sphinxincludegraphics{{articles_list}.png}

Vous pouvez ajouter, modifier ou supprimer un article. La suppression n’est pas possible si l’article est utilisé dans une facture.

A chaque article, vous devez définir un code comptable d’imputation pour la génération d’écritures automatique.
\begin{description}
\item[{Le champ \sphinxstyleemphasis{stockable} permet de définir si vous voulez gérer une gestion de stock de cet article:}] \leavevmode\begin{itemize}
\item {} \begin{description}
\item[{non stockable}] \leavevmode
Article sans gestion de stock, comme par exemple des articles de service.

\end{description}

\item {} \begin{description}
\item[{stockable}] \leavevmode
Article stockable et facturable.

\end{description}

\item {} \begin{description}
\item[{stockable \& non vendable}] \leavevmode
Article stockable non proposable à la vente.
Utile pour suivre des stocks de matériel interne.

\end{description}

\end{itemize}

\end{description}

De plus, dans le cas où vous réalisez des factures avec TVA, vous devrez préciser, pour chaque articles, le taux de taxe à appliquer.

L’onglet \sphinxstyleemphasis{Fournisseur} permet d’identifier des références fournisseurs pour simplifier leur commande ou leur référencement.

Le champ \sphinxstyleemphasis{code d’imputation comptable} permet d’associer à cet article une configuration comptable (voir \sphinxstyleemphasis{Configuration et paramétrage})


\subsection{La facture avec TVA}
\label{\detokenize{invoice/articles:la-facture-avec-tva}}
Si vous êtes soumis à la TVA, l’édition de la facture change un peu

En plus de préciser si les articles sont en montant HT ou TTC, vous avez en bas de l’écran le total de la facture hors\sphinxhyphen{}taxe, taxes comprises ainsi que le montant total des taxes.

De plus, dans l’impression de la facture, un sous\sphinxhyphen{}détail des taxes apparait par taux de TVA.


\subsection{Import d’articles}
\label{\detokenize{invoice/articles:import-d-articles}}
Depuis le menu \sphinxstyleemphasis{Administration/Modules (conf.)/Import d’article}, vous avez la possibilité d’importer des articles en lot depuis un fichier CSV.

Une fois avoir sélectionné votre fichier CVS et les information de son format,
vous serez ammené à associer les champs d’un article aux colonnes de votre documents (la première ligne de votre document doit décrire la nature de chaque colonne).

Vous pouvez alors contrôler vos données avant de les validés.


\section{Création de facture}
\label{\detokenize{invoice/create_bill:creation-de-facture}}\label{\detokenize{invoice/create_bill::doc}}

\subsection{Création}
\label{\detokenize{invoice/create_bill:creation}}
Depuis le menu \sphinxstyleemphasis{Facturier/Facture} vous pouvez éditer ou ajouter une nouvelle facture.

Commencez par définir le type de document (devis, facture, reçu ou avoir) que vous souhaitez créer ainsi que la date d’émission et un commentaire qui figurera dessus.

Dans cette facture, vous devez préciser le client associé, c’est à dire le tiers comptable imputable de l’opération.
\begin{quote}

\noindent\sphinxincludegraphics{{bill_edit}.png}
\end{quote}

Ensuite ajoutez ou enlevez autant d’articles que vous le désirez.
\begin{quote}

\noindent\sphinxincludegraphics{{add_article}.png}
\end{quote}

Par défaut, vous obtenez la désignation et le prix par défaut de l’article sélectionné, mais l’ensemble est modifiable. Vous pouvez choisir aussi l’article divers: aucune information par défaut n’est alors proposé.

Si l’article a été défini comme \sphinxstyleemphasis{stockable}, vous devrez en plus préciser depuis quel lieu de stockage il sera sortie.
Il n’est bien sur pas possible de vendre plus d’article stockable que l’on possède dans le stock.


\subsection{Changement d’état}
\label{\detokenize{invoice/create_bill:changement-d-etat}}
Depuis le menu \sphinxstyleemphasis{Facturier/Facture} vous pouvez consulter les factures en cours, validé ou fini.

Un devis, une facture, un reçu ou un avoir dans l’état « en cours » est un document en cours de conception et il n’est pas encore envoyé au client.

Depuis la fiche du document, vous pouvez le valider: il devient alors imprimable et non modifiable.

Dans ces deux cas, une écriture comptable est alors automatiquement générée.

Un devis validé peut facilement être transformé en facture dans le cas de son acceptation par votre client. La facture ainsi créé se retrouve alors dans l’état « en cours » pour vous permettre de la réajuster.

Une fois qu’une facture (ou un avoir) est considéré comme terminée (c’est à dire réglée ou définie comme pertes et profits), vous pouvez définir son état à «fini».

Depuis une facture « fini », il vous est possible de créer un avoir correspondant à l’état « en cours ». Cette fonctionnalité vous sera utile si vous êtes amené à rembourser un client d’un bien ou un service précédemment facturé.

Si une facture contiens des articles \sphinxstyleemphasis{stockable}, un bordereau de sortie est automatiquement générer pour correspondre à cette vente.
La situation du stock est alors mise à jour automatiquement.


\subsection{Impression}
\label{\detokenize{invoice/create_bill:impression}}
Depuis la fiche d’un document (devis, facture, reçu ou avoir) vous pouvez à tout moment imprimer ou réimprimer celui\sphinxhyphen{}ci s’il n’est pas à l’état «en cours».

Pour les factures, les reçus ou les avoirs, une sauvegarde officielle (d’après le modèle d’impression par défaut) est automatiquement sauvegardée dans le gestionnaire de documents au moment de la validation.
Lorsque vous voulez imprimer le justificatif, on vous propose alors par défaut de télécharger ce document sauvegardé.
Vous pouvez régénérer un nouveau PDF, par exemple avec un autre modèle d’impression. Par contre celui\sphinxhyphen{}ci comportera la mention « duplicata » en filigrane.


\subsection{Paiement}
\label{\detokenize{invoice/create_bill:paiement}}
Si ceux\sphinxhyphen{}ci sont configurés (menu \sphinxstyleemphasis{Administration/Modules (conf.)/Configuration du règlement}), vous pouvez consulter les moyens de paiement d’une facture, d’un reçu ou d’un devis.
Si vous l’envoyez par courriel, vous pouvez également les faire apparaitre dans votre message.

Dans le cas d’un paiement via PayPal, si votre \_Diacamma\_ est accessible par internet, le logiciel sera automatiquement notifié du règlement.
Dans le cas d’un devis, celui\sphinxhyphen{}ci sera automatiquement archivé et une facture équivalente sera générée.
Un nouveau réglement sera ajouté dans votre facture.

Dans l’écran \sphinxstyleemphasis{Comptabilité/Transactions bancaires}, vous pouvez consulté précisement la notification reçu de PayPal.
En cas d’état « échec », la raison est alors précisé: il vous faudra manuellement vérifier votre compte PayPal et rétablir l’éventuellement paiment erroné manuellement.


\section{Gestion de stock}
\label{\detokenize{invoice/stock:gestion-de-stock}}\label{\detokenize{invoice/stock::doc}}

\subsection{Bordereau de stockage}
\label{\detokenize{invoice/stock:bordereau-de-stockage}}
Depuis le menu \sphinxstyleemphasis{Facturier/Stockage/Bordereau de stockage} vous pouvez créer des bordereaux de reception ou sortie d’article stockable.

Pour cela, vous devez préciser un lieu de stockage qui sera impacté par le mouvement de stock (configurable depuis \sphinxstyleemphasis{Administration/Modules (conf.)/Configuration du facturier})
Dans le cas d’une réception, vous pouvez optionnellement préciser une référence de fournisseur (raison social, facture)
Ajoutez alors les articles ainsi que la valeur du mouvement.
Pour un borderaux de sortie, il n’est pas possible de saisir un montant suppérieur au stock courrant.


\subsection{Situation}
\label{\detokenize{invoice/stock:situation}}
Depuis le menu \sphinxstyleemphasis{Facturier/Stockage/Situation} vous pouvez consulter, pour chaque lieu de stockage, la quantité de chaque article géré.

Le bouton \sphinxstyleemphasis{imprimer} permet de sortir un rapport au format PDF ou CSV (importable via un tableur)

Depuis la fiche d’un article, vous pouvez également consulter sa situation de stockage pour chacun des lieux de stockage défini.


\subsection{Historique}
\label{\detokenize{invoice/stock:historique}}
Depuis le menu \sphinxstyleemphasis{Facturier/Stockage/Situation} vous pouvez consulter, le mouvement des articles en réception et en sortie, pour une période de temps données.

Le bouton \sphinxstyleemphasis{imprimer} permet également le même type de rapport que précédement.


\section{Statistiques des ventes}
\label{\detokenize{invoice/statistics:statistiques-des-ventes}}\label{\detokenize{invoice/statistics::doc}}
\noindent\sphinxincludegraphics{{statistiques}.png}

Vous avez la possibilité d’éditer les statistiques des ventes d’un exercice donné.

Ce document vous donne les sommes et le pourcentage par clients, par articles et par mois.
Il est possible également d’avoir la répartition des réglements par mode.

Via le bouton \sphinxstyleemphasis{Contrôler}, il vous est possible d’avoir un contôle de cohérence entre vos montants vendu et vos lignes de comptabilités.
Cela vous permet de vérifier si une correction spécifique aurait desynchronisé financièrement les deux outils.


\section{Configuration et paramétrage}
\label{\detokenize{invoice/configuration:configuration-et-parametrage}}\label{\detokenize{invoice/configuration::doc}}

\subsection{Catégories}
\label{\detokenize{invoice/configuration:categories}}
Le menu \sphinxstyleemphasis{Administration/Modules (conf.)/Configuration commercial du facturier}, onglet \sphinxstyleemphasis{Catégorie}

Vous pouvez définir des catégories afin de classer vos articles.
Chaque article peut être associé à plusieurs catégories.


\subsection{Champ personnalisé}
\label{\detokenize{invoice/configuration:champ-personnalise}}
Le menu \sphinxstyleemphasis{Administration/Modules (conf.)/Configuration commercial du facturier}, onglet \sphinxstyleemphasis{Champ personnalisé}

Comme pour les contacts, vous pouvez ici définir des champs personnalisés.


\subsection{Lieu de stockage}
\label{\detokenize{invoice/configuration:lieu-de-stockage}}
Le menu \sphinxstyleemphasis{Administration/Modules (conf.)/Configuration commercial du facturier}, onglet \sphinxstyleemphasis{Lieu de stockage}

Si vous voulez gérer une centrale d’achat, vous pouvez ici définir les différents espace de vos articles stockables.


\subsection{Réduction automatique}
\label{\detokenize{invoice/configuration:reduction-automatique}}
Le menu \sphinxstyleemphasis{Administration/Modules (conf.)/Configuration commercial du facturier}, onglet \sphinxstyleemphasis{Réduction automatique}
\begin{description}
\item[{Ce tableau de gestion de réductions comporte les champs suivants:}] \leavevmode\begin{itemize}
\item {} 
La catégorie d’article impacté.

\item {} 
Le type de réduction: en valeur, en pourcentage, en pourcentage global déjà vendu.

\item {} 
Le montant de la réduction (en valeur ou pourcentage suivant le type).

\item {} 
Le nombre d’occurrence déclenchant la réduction.

\item {} 
Un critère de filtrage du tiers à qui cette réduction s’applique. (Optionnel)

\end{itemize}

\end{description}

Au moment d’ajout d’un article dans une facture, si le client de cette facture et ce nouvel article répond à aux critères d’une réduction,
celle\sphinxhyphen{}ci s’applique alors automatiquement dans la facture.
Si plusieurs réductions remplissent leurs conditions, c’est la réduction octroyant la plus grande réduction qui sera utilisé.
Un article peut se retrouver vendu gratuitement, mais jamais négativement (qui reviendrait à un remboursement)
Ce mécanisme sera également appliqué lors de la création automatique des factures (cotisation, participation à un événement)
Ce mécanisme vérifie le critère que pour des opérations réalisés sur l’exercice financier courant de l’association (les réductions ne se cumule pas d’une année à l’autre)


\subsection{Codes d’imputations comptable}
\label{\detokenize{invoice/configuration:codes-d-imputations-comptable}}
Menu \sphinxstyleemphasis{Administration/Modules (conf.)/Configuration financière du facturier}, onglet \sphinxstyleemphasis{Codes d’imputations comptable}
\begin{description}
\item[{Un « Code d’imputation comptable » contiens:}] \leavevmode\begin{itemize}
\item {} 
un code comptable de vente

\item {} 
un code analytique (optionnel)

\end{itemize}

\end{description}

Chaque article peut être associé à un code d’imputation comptable (si non précisé, l’article n’est pas vendable).
Ce mécanisme permet de centraliser à un seul endroit les configurations comptables des articles.
Au changement d’exercice, si ces configurations doivent changées, il est plus simple de modifier cette configuration que l’ensemble des articles.


\subsection{Codes comptables par défaut}
\label{\detokenize{invoice/configuration:codes-comptables-par-defaut}}
Menu \sphinxstyleemphasis{Administration/Modules (conf.)/Configuration financière du facturier}, onglet \sphinxstyleemphasis{Paramètres}

Ce module est intimement lié au module de gestion comptable, un certain nombre de codes comptables par défaut sont nécessaires.

Pour pouvoir générer les écritures comptables correspondants aux factures saisies avec des articles non référencés, vous devez préciser le code comptable de vente (classe 7) lié a ce type de d’article. Par défaut, le code comptable de vente de service est défini.

Pour réaliser une réduction sur un article, vous devez préciser le code comptable de vente à imputer de cette réduction.
Dans le cas de règlement en liquide, il vous faut préciser le code comptable de banque associé à votre caisse.


\subsection{La configuration de la TVA}
\label{\detokenize{invoice/configuration:la-configuration-de-la-tva}}
Menu \sphinxstyleemphasis{Administration/Modules (conf.)/Configuration financière du facturier}, onglet \sphinxstyleemphasis{TVA}

Vous pouvez complètement configurer la gestion de votre soumission à la TVA.

\noindent\sphinxincludegraphics{{vat}.png}

Pour commencer, vous devez définir les modalités de soumission en sélectionnant votre mode d’application:
\begin{itemize}
\item {} \begin{description}
\item[{TVA non applicable}] \leavevmode
Vous n’êtes pas soumis à la TVA. L’ensemble de vos factures sont réalisées hors\sphinxhyphen{}taxe.

\end{description}

\item {} \begin{description}
\item[{Prix HT}] \leavevmode
Vous êtes soumis à la TVA. Vous faites le choix d’éditer vos factures avec les montants des articles en hors\sphinxhyphen{}taxe.

\end{description}

\item {} \begin{description}
\item[{Prix TTC}] \leavevmode
Vous êtes soumis à la TVA. Vous faites le choix d’éditer vos factures avec les montants des articles toutes taxes comprises.

\end{description}

\end{itemize}

Précisez également le compte comptable d’imputation de ces taxes.

Définissez l’ensemble des taux de taxes auxquels vos ventes sont soumises. La taxe par défaut sera celle appliquée à l’article libre (sans référence).


\chapter{Diacamma règlement}
\label{\detokenize{payoff/index:diacamma-reglement}}\label{\detokenize{payoff/index::doc}}
Aide relative aux fonctionnalités de gestion des payements.


\section{Règlement}
\label{\detokenize{payoff/payoff:reglement}}\label{\detokenize{payoff/payoff::doc}}
Depuis un module tel que la facturation, il vous est possible de gérer leur règlement.

Depuis la fiche du document, cliquez sur «ajouter» un paiement.
\begin{quote}

\noindent\sphinxincludegraphics{{payoff}.png}
\end{quote}

Vous pouvez alors saisir le mode de paiement de votre client ainsi que le compte bancaire à imputer de ce mouvement financier.

Dans la facture, vous pouvez consulter en plus de son montant total, la somme versée ainsi que le résidu de paiement à effectuer.

Chaque règlement génère automatiquement une écriture comptable dans le journal d’encaissement.

Il est aussi possible d’effectuer un seul règlement sur plusieurs document financier (comme les factures). Pour cela sélectionnez dans la liste des éléments « valides » celles que vous souhaitez et cliquez sur Réglement.
\begin{quote}

\noindent\sphinxincludegraphics{{multi-payoff}.png}
\end{quote}

Suivant le type de document sur lequel ce paiement est associé, vous pouvez avoir plusieurs modes de répartition:
\begin{itemize}
\item {} 
Par date
Ce paiement est d’abort ventilé sur le document financier le plus ancien, puis le suivant, etc.

\item {} 
Par prorata
Ce paiement multiple sera automatique ventilé sur document financier au prorata de leur montant.

\end{itemize}

Dans tout les cas, une seule écriture comptable d’encaissement sera alors réalisée.


\section{Dépôt de chèques}
\label{\detokenize{payoff/deposit:depot-de-cheques}}\label{\detokenize{payoff/deposit::doc}}
Depuis le menu \sphinxstyleemphasis{Comptabilité/Dépôt de chèques}, vous pouvez ajouter et consulter des bordereaux de chèques.
\begin{quote}

\noindent\sphinxincludegraphics{{depositlist}.png}
\end{quote}

Dans une fiche de bordereau, vous pouvez sélectionner les règlements effectués par chèques dans vos différentes factures.
Cela vous constitue une liasse de chèques que vous pourrez déposer à votre agence bancaire.
Une fois réalisée, clôturez la sélection définie.
\begin{quote}

\noindent\sphinxincludegraphics{{deposititem}.png}
\end{quote}

Vous pouvez alors imprimer le bordereau de remise de chèques que vous pouvez joindre à votre liasse lors du dépôt de celle\sphinxhyphen{}ci.
De plus, une fois que votre bordereau apparaît sur votre relevé de compte, vous pouvez valider l’ensemble de vos écritures comptables depuis la fiche elle\sphinxhyphen{}même.


\section{Configuration}
\label{\detokenize{payoff/config:configuration}}\label{\detokenize{payoff/config::doc}}
Le menu \sphinxstyleemphasis{Administration/Modules (conf.)/Configuration du règlement} vous permet quelques configurations pour votre structure.


\subsection{Compte bancaire}
\label{\detokenize{payoff/config:compte-bancaire}}
Dans cet écran, vous avez la possibilité d’enregistrer vos différents comptes bancaires que vous possédez.
Pour chacun, vous pouvez saisir l’intégralité des informations figurant sur un RIB.
Cela vous permettra d’éditer un résumé complet de vos dépôts de chèques.


\subsection{Moyen de paiement}
\label{\detokenize{payoff/config:moyen-de-paiement}}
Vous pouvez ici préciser les moyens de paiement que vous supportez.
Actuellement, 3 moyens de paiement sont pris en compte par \sphinxstyleemphasis{Diacamma}
\begin{itemize}
\item {} 
Le virement bancaire

\item {} 
Le chèque

\item {} 
Le paiement PayPal

\end{itemize}

Pour chacun d’entre eux, vous devez préciser les paramètres relatifs.

C’est moyen de paiement peuvent être utilisé pour vos débiteurs afin de régler par un de ses moyens ce qu’ils vous doivent.

Dans le cas de PayPal, si votre \sphinxstyleemphasis{Diacamma} est accessible par internet, le logiciel peux être notifié directement du paiement et ajouter un réglement correspondant directement dans votre logiciel.
Il est conseillé, dans ce cas, de cocher le champ \sphinxstyleemphasis{avec contrôle}: le lien de paiement présenter dans un courriel redirigera alors en premier sur votre \sphinxstyleemphasis{Diacamma} afin de vérifier que cet élément financier est toujours d’actualité.


\subsection{Paramètres}
\label{\detokenize{payoff/config:parametres}}
2 Paramètres actuellements:
\begin{itemize}
\item {} 
compte de caisse: indique le code comptable à imputer pour les règlements en espèce.

\item {} 
compte de frais bancaire: prècise un code comptable pour imputer directement, suite à un règlement, des frais bancaires inhérent à ce règlement.

\end{itemize}

Un ligne d’écriture est alors ajouté directement à l’écriture comptable correspondant.
Si ce code est vide, aucun frais bancaire ne vous sera demandé.


\chapter{Lucterios contacts}
\label{\detokenize{contacts/index:lucterios-contacts}}\label{\detokenize{contacts/index::doc}}
Aide relative aux fonctionnalités de gestion de contacts moraux ou physiques.


\section{Les contact physiques}
\label{\detokenize{contacts/individual:les-contact-physiques}}\label{\detokenize{contacts/individual::doc}}
Un contact physique est une personne, homme ou femme, à mémoriser.


\subsection{Liste de vos contacts physiques}
\label{\detokenize{contacts/individual:liste-de-vos-contacts-physiques}}
Le menu \sphinxstyleemphasis{Bureautique/Adresses et Contacts/Personnes Physiques} vous permet de consulter la liste des personnes que vous avez déjà enregistrées. Étant donné que la liste peut devenir importante, il est possible de filtrer les personnes par leur nom.

Depuis cet écran, vous avez aussi la possibilité d’imprimer la liste des personnes.

\noindent\sphinxincludegraphics{{ListIndividual}.png}


\subsection{Edition d’un contact physique}
\label{\detokenize{contacts/individual:edition-d-un-contact-physique}}
Depuis la liste précédente, vous avez la possibilité d’ajouter une nouvelle personne. Vous pouvez ré\sphinxhyphen{}éditer cette fiche depuis sa visualisation.

\noindent\sphinxincludegraphics{{EditIndividual}.png}


\subsection{Visualisation d’un contact physique}
\label{\detokenize{contacts/individual:visualisation-d-un-contact-physique}}
Depuis la liste des personnes physiques, vous avez la possibilité de visualiser une personne.

Cela vous permettra de consulter la fiche d’une personne précédemment enregistrée dans votre base. Vous pouvez modifier cette fiche ou l’imprimer. Vous pouvez également lui donner un alias de connexion à l’application associé à un droit d’accès (voir Les utilisateurs). Si cette personne n’est pas référencée dans d’autres enregistrements de l’application, vous avez la possibilité de la supprimer.

\noindent\sphinxincludegraphics{{ShowIndividual}.png}


\subsection{Recherche d’un contact physique}
\label{\detokenize{contacts/individual:recherche-d-un-contact-physique}}
Le menu Bureautique/Adresses et Contacts/Recherche de personne physique de personne physique vous permet de définir un critère de recherche sur une personne physique.

Une fois validé, l’outil va rechercher dans la base toutes les personnes correspondantes à ces critères. Vous pourrez alors imprimer cette liste ou en visualiser/modifier une fiche.

\noindent\sphinxincludegraphics{{FindIndividual}.png}


\section{Les contacts moraux}
\label{\detokenize{contacts/legal_entity:les-contacts-moraux}}\label{\detokenize{contacts/legal_entity::doc}}
Un contact moral est une structure ou d’une organisation de personne (entreprise, association, administration, …), à mémoriser.


\subsection{Liste de vos contacts moraux}
\label{\detokenize{contacts/legal_entity:liste-de-vos-contacts-moraux}}
Le menu \sphinxstyleemphasis{Bureautique/Adresses et Contacts/Personnes morales} vous permet de consulter la liste des structures que vous avez déjà enregistrées. Chaque contact moral est associé à une catégorie. Dans cette liste, vous consultez vos structures filtrées par ces catégories.

Depuis cet écran, vous avez aussi la possibilité d’imprimer la liste des structures.

\noindent\sphinxincludegraphics{{ListLegalEntity}.png}


\subsection{Edition d’un contact moral}
\label{\detokenize{contacts/legal_entity:edition-d-un-contact-moral}}
Depuis la liste précédente, vous avez la possibilité de créer une nouvelle structure. Vous pouvez ré\sphinxhyphen{}éditer cette fiche depuis sa visualisation.

\noindent\sphinxincludegraphics{{EditLegalEntity}.png}


\subsection{Visualisation d’un contact moral}
\label{\detokenize{contacts/legal_entity:visualisation-d-un-contact-moral}}
Depuis la liste des personnes morales, vous avez la possibilité de visualiser une structure.

Cela vous permettra de consulter la fiche d’une structure précédemment entrée dans votre base. Vous pouvez modifier cette fiche ou l’imprimer. Si cette personne n’est pas référencée dans d’autre enregistrement de l’application, vous avez la possibilité de la supprimer.

\noindent\sphinxincludegraphics{{ShowLegalEntity}.png}


\subsection{Responsables d’un contact moral}
\label{\detokenize{contacts/legal_entity:responsables-d-un-contact-moral}}
Vous avez la possibilité d’associer une personne physique à votre structure.

Choisissez le nouveau responsable: si la personne n’existe pas dans votre base, vous aurez la possibilité de la créer. Vous pourrez également ajouter une fonction à un responsable défini.

\noindent\sphinxincludegraphics{{ResponsabilityLegalEntity}.png}


\subsection{Recherche d’un contact moral}
\label{\detokenize{contacts/legal_entity:recherche-d-un-contact-moral}}
Le menu \sphinxstyleemphasis{Bureautique/Adresses et Contacts/Recherche de personne morale} vous permet de définir un critère de recherche sur une structure morale.

\noindent\sphinxincludegraphics{{FindLegalEntity}.png}


\section{Configuration et paramétrage}
\label{\detokenize{contacts/configuration:configuration-et-parametrage}}\label{\detokenize{contacts/configuration::doc}}
Dans le menu \sphinxstyleemphasis{Administration/Modules (conf.)} vous avez à votre disposition des outils pour configurer la gestion des contacts.


\subsection{Configuration des contacts}
\label{\detokenize{contacts/configuration:configuration-des-contacts}}
Dans cet écran, vous avez la possibilité de créer ou de modifier une définition de fonction, ou responsabilité, pour associer une personne physique à une structure morale. Vous pouvez créer ou modifier une catégorie de structure morale pour vous aider dans la classification de vos contacts moraux.

Il se peut que vous ayez besoin de préciser des informations supplémentaires pour vos différents contacts. Vous avez ici la possibilité d’ajouter des champs personnels pour chaque type de contacts. Pour ajouter un champ, vous devez simplement donner son titre ainsi que définir son type et éventuellement les compléments nécessaires.
5 types are possibles:
\begin{itemize}
\item {} 
chaîne de texte

\item {} 
nombre entier

\item {} 
nombre à virgule (réel)

\item {} 
valeur Oui/Non (booléen)

\item {} 
choix dans une liste (énumération)

\end{itemize}

Dans le cas de l’énumération, vous devez définir la liste des valeurs possibles (mots) séparées par un point\sphinxhyphen{}virgule.


\subsection{Codes postaux/villes}
\label{\detokenize{contacts/configuration:codes-postaux-villes}}
Cela peux vous aider dans votre saisi de contact, l’outil va automatiquement rechercher la ville (ou les villes) associée(s) avec le code postal que vous entrerez.
Dans cet écran, vous pouvez ajouter des codes postaux manquants.
Par défaut, les codes postaux français et suisses sont insérés.


\chapter{Lucterios courier}
\label{\detokenize{mailing/index:lucterios-courier}}\label{\detokenize{mailing/index::doc}}
Aide relative aux fonctionnalités de courier et publipostage.


\section{Publipostage}
\label{\detokenize{mailing/mailing:publipostage}}\label{\detokenize{mailing/mailing::doc}}
Depuis le menu \sphinxstyleemphasis{Bureatique/Publipostage/Message} vous avez la possibilité de créer un courier de publipostage.


\subsection{Création d’un message}
\label{\detokenize{mailing/mailing:creation-d-un-message}}
Une fois votre message rédigé, vous pouvez lui associé des requetes de destinataires.
C’est requetes de recherches, similaire à celle des outils de recherche de contacts, ne seront évaluées qu’au moment de la génération du courier.
Ainsi, même un contact dernièrement ajouté ou modifié pourra être impacté par ce message.

Il est également possible d’ajouter à votre message un ou plusieurs documents, sauvés dans le \sphinxstyleemphasis{gestionnaire de documentation}.
Ces documents seront transmis en pièces\sphinxhyphen{}jointes dans l’envoie par courriel.

L’option \sphinxstyleemphasis{document(s) ajouté(s) via liens dans le message} permet d’ajouter un ensemble de liens partagés vers vos documents (et non plus des pièces jointes).
Cela permet de gérer des documents de taille importante ou qui risqueraient d’être supprimer par certain gestionnaire de courriel.

\noindent\sphinxincludegraphics{{mailing}.png}


\subsection{Validation \& transmission}
\label{\detokenize{mailing/mailing:validation-transmission}}\begin{description}
\item[{Une fois le message validé vous pouvez:}] \leavevmode\begin{itemize}
\item {} 
Soit généré une sortie PDF de l’ensemble des lettres à envoyer personnalisé avec l’entête de chaque contact

\item {} 
Soit envoyé par courriel si votre configuration est valide. Bien sur, dans ce cas, seul les contacts possédant une adresse seront impacté par cet envoie.

\end{itemize}

\end{description}

De plus, dans le cas d’un envoie par courriel, vous pouvez consulter un rapport de transmission.
Celui\sphinxhyphen{}ci vous indique les courriels envoyés, leur éventuel erreur d’acheminement.

Si votre logiciel est accessible depuis internet, vous pouvez également consulter le nombre de fois que le destinataire à consulter ce message.
Ce mécanisme se base sur l’acceptation, par votre destinataire des images distantes présentent dans le message.

\noindent\sphinxincludegraphics{{transmission}.png}


\section{Configuration du couriel}
\label{\detokenize{mailing/configuration:configuration-du-couriel}}\label{\detokenize{mailing/configuration::doc}}
Vous pouvez configurer ici des réglages pour l’envoi de couriel.

Le serveur SMTP permettra au logiciel d’envoyer directement des messages à vos contacts.
Configurez donc ici les règlages de votre serveur.
Vous pouvez également préciser un \sphinxstyleemphasis{Fichier privé DKIM} et \sphinxstyleemphasis{Sélecteur DKIM} afin de signer vos envoies de courriel.
Les paramètres \sphinxstyleemphasis{durée (en min) d’un lot de courriel} et \sphinxstyleemphasis{nombre de courriels par lot} sont utilisés pour l’envoie des messages en publipostage.

Un bouton \sphinxstyleemphasis{Envoyer} permet de tester vos règlages en envoyant un courriel de test à un destinataire choisi.
Il existe des outils permettant de vérifier si vos messages respectent des règles afin d’éviter d’être considérer comme des “pourriel”.
En autre, l’outil \sphinxurl{https://www.mail-tester.com} (gratuit jusqu’à 3 fois par jour) vous permet, en envoyant un message à l’adresse précisée, de vous établir une note de confiance.

Vous pouvez, entre autre, envoyer d’un nouveau mot de passe de connexion.
N’oubliez pas alors de préciser un petit message d’explication via le paramètres \sphinxstyleemphasis{Message de confirmation de connexion}.


\chapter{Lucterios documents}
\label{\detokenize{documents/index:lucterios-documents}}\label{\detokenize{documents/index::doc}}
Aide relative aux fonctionnalités de gestion documentaire.


\section{Fichiers partagés}
\label{\detokenize{documents/shared_document:fichiers-partages}}\label{\detokenize{documents/shared_document::doc}}

\subsection{Liste des documents}
\label{\detokenize{documents/shared_document:liste-des-documents}}
Le menu \sphinxstyleemphasis{Bureautique/Gestion documentaire/Documents} vous permet de consulter la liste des fichiers que vous avez déjà enregistrés. Pour vous aider à retrouver vos documents, la liste est classifiée par un ensemble de dossiers et de sous\sphinxhyphen{}dossiers et une description vous donne un petit résumé.

Vous avez aussi la possibilité d’ajouter un sous\sphinxhyphen{}dossier ou de modifier les propriétés du dossier courant.

\noindent\sphinxincludegraphics{{listdoc}.png}

Suivant vos permissions, vous pouvez extraire votre fichier pour le consulter, le modifier et éventuellement ré\sphinxhyphen{}injecter vos corrections.

De plus, l’outil mémorisera l’utilisateur et la date de création du document ainsi que les informations relatives à la dernière modification.

\noindent\sphinxincludegraphics{{showdoc}.png}

Depuis la fiche du document, il vous est possible d’activer un lien de téléchargement.
Ce lien web peut être transmis à une personne tiers, n’ayant aucun droit d’accès à votre logiciel, afin de télécharger le document.
\sphinxstylestrong{Attention:} Votre instance doit être accessible sur internet pour que le lien puisse fonctionner depuis n’importe où.


\subsection{Recherche de documents}
\label{\detokenize{documents/shared_document:recherche-de-documents}}
Le menu \sphinxstyleemphasis{Bureautique/Gestion documentaire/Recherche de document} vous permet de définir un critère de recherche sur un document.

Une fois validé, l’outil va rechercher dans la base toutes les fichiers correspondants à ces critères.


\section{Editeur de documents}
\label{\detokenize{documents/editor:editeur-de-documents}}\label{\detokenize{documents/editor::doc}}
Il est possible de configurer l’outil afin de pouvoir éditer certain document directement via l’interface « en ligne ».

Des outils d’édition, libres et gratuits, sont actuellement configurable afin de les utiliser pour consulter et modifier des documents.
\_Note:\_ Ces outils sont gérés par des équipes complètement différentes, il se peux que certain de leur comportement ne correspondent pas à vos attentes.


\subsection{etherpad}
\label{\detokenize{documents/editor:etherpad}}
Editeur pour document textuel.
\begin{description}
\item[{Site Web}] \leavevmode
\sphinxurl{https://etherpad.org/}

\item[{Installation}] \leavevmode
Le tutoriel de framasoft explique bien comment l’installer
\sphinxurl{https://framacloud.org/fr/cultiver-son-jardin/etherpad.html}

\item[{Configurer}] \leavevmode
Editer le fichier « settings.py » contenu dans le répertoire de votre instance.
Ajouter et adapter la ligne ci\sphinxhyphen{}dessous:
\begin{itemize}
\item {} 
url : adresse d’accès d’etherpad

\item {} 
apikey : contenu de la clef de sécurité (fichier APIKEY.txt contenu dans l’installation d’etherpad)

\end{itemize}

\end{description}

\begin{sphinxVerbatim}[commandchars=\\\{\}]
\PYG{c+c1}{\PYGZsh{} extra}
\PYG{n}{ETHERPAD} \PYG{o}{=} \PYG{p}{\PYGZob{}}\PYG{l+s+s1}{\PYGZsq{}}\PYG{l+s+s1}{url}\PYG{l+s+s1}{\PYGZsq{}}\PYG{p}{:} \PYG{l+s+s1}{\PYGZsq{}}\PYG{l+s+s1}{http://localhost:9001}\PYG{l+s+s1}{\PYGZsq{}}\PYG{p}{,} \PYG{l+s+s1}{\PYGZsq{}}\PYG{l+s+s1}{apikey}\PYG{l+s+s1}{\PYGZsq{}}\PYG{p}{:} \PYG{l+s+s1}{\PYGZsq{}}\PYG{l+s+s1}{jfks5dsdS65lfGHsdSDQ4fsdDG4lklsdq6Gfs4Gsdfos8fs}\PYG{l+s+s1}{\PYGZsq{}}\PYG{p}{\PYGZcb{}}
\end{sphinxVerbatim}
\begin{description}
\item[{Usage}] \leavevmode\begin{description}
\item[{Dans le gestionnaire de document, vous avez plusieurs action qui apparait alors}] \leavevmode\begin{itemize}
\item {} 
Un bouton « + Fichier » vous permettant de créer un document txt ou html

\item {} 
Un bouton « Editeur » pour ouvrir l’éditeur etherpad.

\end{itemize}

\end{description}

\end{description}

\noindent\sphinxincludegraphics{{etherpad}.png}


\subsection{ethercalc}
\label{\detokenize{documents/editor:ethercalc}}
Editeur pour tableau de calcul.
\begin{description}
\item[{Site Web}] \leavevmode
\sphinxurl{https://ethercalc.net/}

\item[{Installation}] \leavevmode
Sur le site de l’éditeur, une petit explication indique comment l’installer.

\item[{Configurer}] \leavevmode
Editer le fichier « settings.py » contenu dans le répertoire de votre instance.
Ajouter et adapter la ligne ci\sphinxhyphen{}dessous:
\begin{itemize}
\item {} 
url : adresse d’accès d’ethercal

\end{itemize}

\end{description}

\begin{sphinxVerbatim}[commandchars=\\\{\}]
\PYG{c+c1}{\PYGZsh{} extra}
\PYG{n}{ETHERCALC} \PYG{o}{=} \PYG{p}{\PYGZob{}}\PYG{l+s+s1}{\PYGZsq{}}\PYG{l+s+s1}{url}\PYG{l+s+s1}{\PYGZsq{}}\PYG{p}{:} \PYG{l+s+s1}{\PYGZsq{}}\PYG{l+s+s1}{http://localhost:8000}\PYG{l+s+s1}{\PYGZsq{}}\PYG{p}{\PYGZcb{}}
\end{sphinxVerbatim}
\begin{description}
\item[{Usage}] \leavevmode\begin{description}
\item[{Dans le gestionnaire de document, vous avez plusieurs action qui apparait alors}] \leavevmode\begin{itemize}
\item {} 
Un bouton « + Fichier » vous permettant de créer un document csv, ods ou xmlx

\item {} 
Un bouton « Editeur » pour ouvrir l’éditeur ethercalc.

\end{itemize}

\end{description}

\end{description}

\noindent\sphinxincludegraphics{{ethercalc}.png}


\section{Configuration}
\label{\detokenize{documents/configuration:configuration}}\label{\detokenize{documents/configuration::doc}}
Dans le menu \sphinxstyleemphasis{Administration/Module (conf)/Dossiers} vous avez à votre disposition un ensemble d’outils pour configurer la gestion documentaire.


\subsection{Dossiers}
\label{\detokenize{documents/configuration:dossiers}}
Dans cet écran vous avez la possibilité de créer ou de modifier des dossiers de classement documentaire.

\noindent\sphinxincludegraphics{{configuration}.png}

En associant judicieusement un dossier comme sous\sphinxhyphen{}dossier d’un parent, vous pouvez vous définir une arborescence de classement.

Vous associez à chaque dossier un ensemble de groupe de droits pour la visualisation et la modification des fichiers. Seuls les utilisateurs appartenant aux groupes de visualisation pourront consulter les documents de cette catégorie. Seuls les utilisateurs appartenant aux groupes de modification pourront corriger les documents de cette catégorie.


\chapter{Coeur Lucterios}
\label{\detokenize{CORE/index:coeur-lucterios}}\label{\detokenize{CORE/index::doc}}
Aide relative aux fonctionnalités générales de cet outil de gestion.


\section{Mot de passe}
\label{\detokenize{CORE/password:mot-de-passe}}\label{\detokenize{CORE/password::doc}}
Le menu \sphinxtitleref{Général/Mot de passe} vous permet de changer le mot de passe d’accès de l’utilisateur courant.

\noindent\sphinxincludegraphics{{password}.png}

Pour plus de sécurité, nous vous conseillons d’utiliser un mot de passe comprenant des lettres et des chiffres et ne constituant pas un mot compréhensible.


\section{Les groupes}
\label{\detokenize{CORE/groups:les-groupes}}\label{\detokenize{CORE/groups::doc}}
Le menu \sphinxtitleref{Administration/Gestion des Droits/Les groupes} vous permet de créer, modifier ou supprimer un groupe de droits.

\noindent\sphinxincludegraphics{{group}.png}

Un groupe de droits réunit un ensemble d’autorisations aux actions de l’application.

\noindent\sphinxincludegraphics{{group_modify}.png}


\section{Les utilisateurs}
\label{\detokenize{CORE/users:les-utilisateurs}}\label{\detokenize{CORE/users::doc}}
Le menu \sphinxtitleref{Administration/Gestion des Droits/Les utilisateurs} vous
permet de créer, modifier ou désactiver un utilisateur de l’application. Un
utilisateur définit un droit de connexion au logiciel.

\noindent\sphinxincludegraphics{{users}.png}

Depuis cette liste, vous pouvez créer ou modifier l’utilisateur: son
alias, son nom et son mot de passe. A cela, vous lui ajouter des groupes et
des permissions suplémentaires éventuelles afin de définir son niveau
d’accès au logiciel. Vous pouvez aussi désactiver un utilisateur pour lui
interdire l’accès à l’application.

\noindent\sphinxincludegraphics{{user_info}.png}

\noindent\sphinxincludegraphics{{user_permissions}.png}


\section{L’architecture du logiciel}
\label{\detokenize{CORE/architecture:l-architecture-du-logiciel}}\label{\detokenize{CORE/architecture::doc}}
Depuis le commencement de ce logiciel, les développeurs ont voulu que cette application puisse avoir une architecture ouverte permettant des évolutions les plus larges.



\renewcommand{\indexname}{Index}
\printindex
\end{document}