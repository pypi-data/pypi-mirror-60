%% Generated by Sphinx.
\def\sphinxdocclass{report}
\documentclass[letterpaper,10pt,french]{sphinxmanual}
\ifdefined\pdfpxdimen
   \let\sphinxpxdimen\pdfpxdimen\else\newdimen\sphinxpxdimen
\fi \sphinxpxdimen=.75bp\relax

\PassOptionsToPackage{warn}{textcomp}
\usepackage[utf8]{inputenc}
\ifdefined\DeclareUnicodeCharacter
% support both utf8 and utf8x syntaxes
  \ifdefined\DeclareUnicodeCharacterAsOptional
    \def\sphinxDUC#1{\DeclareUnicodeCharacter{"#1}}
  \else
    \let\sphinxDUC\DeclareUnicodeCharacter
  \fi
  \sphinxDUC{00A0}{\nobreakspace}
  \sphinxDUC{2500}{\sphinxunichar{2500}}
  \sphinxDUC{2502}{\sphinxunichar{2502}}
  \sphinxDUC{2514}{\sphinxunichar{2514}}
  \sphinxDUC{251C}{\sphinxunichar{251C}}
  \sphinxDUC{2572}{\textbackslash}
\fi
\usepackage{cmap}
\usepackage[T1]{fontenc}
\usepackage{amsmath,amssymb,amstext}
\usepackage{babel}



\usepackage{times}
\expandafter\ifx\csname T@LGR\endcsname\relax
\else
% LGR was declared as font encoding
  \substitutefont{LGR}{\rmdefault}{cmr}
  \substitutefont{LGR}{\sfdefault}{cmss}
  \substitutefont{LGR}{\ttdefault}{cmtt}
\fi
\expandafter\ifx\csname T@X2\endcsname\relax
  \expandafter\ifx\csname T@T2A\endcsname\relax
  \else
  % T2A was declared as font encoding
    \substitutefont{T2A}{\rmdefault}{cmr}
    \substitutefont{T2A}{\sfdefault}{cmss}
    \substitutefont{T2A}{\ttdefault}{cmtt}
  \fi
\else
% X2 was declared as font encoding
  \substitutefont{X2}{\rmdefault}{cmr}
  \substitutefont{X2}{\sfdefault}{cmss}
  \substitutefont{X2}{\ttdefault}{cmtt}
\fi


\usepackage[Sonny]{fncychap}
\ChNameVar{\Large\normalfont\sffamily}
\ChTitleVar{\Large\normalfont\sffamily}
\usepackage{sphinx}

\fvset{fontsize=\small}
\usepackage{geometry}


% Include hyperref last.
\usepackage{hyperref}
% Fix anchor placement for figures with captions.
\usepackage{hypcap}% it must be loaded after hyperref.
% Set up styles of URL: it should be placed after hyperref.
\urlstyle{same}

\usepackage{sphinxmessages}
\setcounter{tocdepth}{3}
\setcounter{secnumdepth}{3}


\title{Lucterios Standard}
\date{janv. 30, 2020}
\release{2.4.0}
\author{sd-libre}
\newcommand{\sphinxlogo}{\vbox{}}
\renewcommand{\releasename}{Version}
\makeindex
\begin{document}

\ifdefined\shorthandoff
  \ifnum\catcode`\=\string=\active\shorthandoff{=}\fi
  \ifnum\catcode`\"=\active\shorthandoff{"}\fi
\fi

\pagestyle{empty}
\sphinxmaketitle
\pagestyle{plain}
\sphinxtableofcontents
\pagestyle{normal}
\phantomsection\label{\detokenize{index::doc}}



\chapter{Lucterios Standard}
\label{\detokenize{standard/index:lucterios-standard}}\label{\detokenize{standard/index::doc}}
Présentation du logiciel Lucterios Standard.


\section{Pour les utilisateurs}
\label{\detokenize{standard/index:pour-les-utilisateurs}}
L’application standard Lucterios vous permet, suite à une installation minimale du système sur votre machine, l’utilisation de modules gestions personnalisés qui répondront à vos besoins.

Les modules les plus faciles à obtenir sont les extensions open\sphinxhyphen{}source développés et maintenus par l’équipe Lusterios ou ajoutés au serveur de mises à jour par d’autres développeurs désireux de partager leur travail. Ils s’installent directement depuis la fonctionnalité de mise à jour inclus dans l’application.

Ces modules seront de plus en plus nombreux au fur et à mesure du temps et des développements. On y compte déjà plusieurs modules: rendez\sphinxhyphen{}vous sur le site de Lucterios pour en savoir plus.

Une autre solution consite à ajouter une adresse de serveur de mise à jours à orientation commercial. Les modules ainsi téléchargés nécessitent parfois une indémnité financière pour les utilisés. Consultez directement les équipes de développement qui réalisent de telles extensions sur leurs modalités.


\section{Pour les developpeurs}
\label{\detokenize{standard/index:pour-les-developpeurs}}
Lucterios est un moteur d’applications client\sphinxhyphen{}serveur, multi\sphinxhyphen{}plateforme et open\sphinxhyphen{}source reposant sur une principe de modules interdépendants.

Si vous le désirez, vous pouvez créer vos propres modules d’extensions pour répondre à vos besoins spécifiques.
Cet outil Lucterios vous permettra de créer un module et l’ensemble des actions associées à vos données: rendez\sphinxhyphen{}vous sur le site de Lucterios pour consulter les tutoriels développeurs.

De la même façon que vous avez pu apprécier des modules open\sphinxhyphen{}sources, vos extensions peuvent peut\sphinxhyphen{}être interessé d’autres personnes.
Contactez nous pour les rendres disponibles sur le serveur de mise à jours.
\begin{quote}

\sphinxhref{mailto:support@lucterios.org}{support@lucterios.org}

\sphinxurl{http://www.lucterios.org}
\end{quote}


\chapter{Lucterios contacts}
\label{\detokenize{contacts/index:lucterios-contacts}}\label{\detokenize{contacts/index::doc}}
Aide relative aux fonctionnalités de gestion de contacts moraux ou physiques.


\section{Les contact physiques}
\label{\detokenize{contacts/individual:les-contact-physiques}}\label{\detokenize{contacts/individual::doc}}
Un contact physique est une personne, homme ou femme, à mémoriser.


\subsection{Liste de vos contacts physiques}
\label{\detokenize{contacts/individual:liste-de-vos-contacts-physiques}}
Le menu \sphinxstyleemphasis{Bureautique/Adresses et Contacts/Personnes Physiques} vous permet de consulter la liste des personnes que vous avez déjà enregistrées. Étant donné que la liste peut devenir importante, il est possible de filtrer les personnes par leur nom.

Depuis cet écran, vous avez aussi la possibilité d’imprimer la liste des personnes.

\noindent\sphinxincludegraphics{{ListIndividual}.png}


\subsection{Edition d’un contact physique}
\label{\detokenize{contacts/individual:edition-d-un-contact-physique}}
Depuis la liste précédente, vous avez la possibilité d’ajouter une nouvelle personne. Vous pouvez ré\sphinxhyphen{}éditer cette fiche depuis sa visualisation.

\noindent\sphinxincludegraphics{{EditIndividual}.png}


\subsection{Visualisation d’un contact physique}
\label{\detokenize{contacts/individual:visualisation-d-un-contact-physique}}
Depuis la liste des personnes physiques, vous avez la possibilité de visualiser une personne.

Cela vous permettra de consulter la fiche d’une personne précédemment enregistrée dans votre base. Vous pouvez modifier cette fiche ou l’imprimer. Vous pouvez également lui donner un alias de connexion à l’application associé à un droit d’accès (voir Les utilisateurs). Si cette personne n’est pas référencée dans d’autres enregistrements de l’application, vous avez la possibilité de la supprimer.

\noindent\sphinxincludegraphics{{ShowIndividual}.png}


\subsection{Recherche d’un contact physique}
\label{\detokenize{contacts/individual:recherche-d-un-contact-physique}}
Le menu Bureautique/Adresses et Contacts/Recherche de personne physique de personne physique vous permet de définir un critère de recherche sur une personne physique.

Une fois validé, l’outil va rechercher dans la base toutes les personnes correspondantes à ces critères. Vous pourrez alors imprimer cette liste ou en visualiser/modifier une fiche.

\noindent\sphinxincludegraphics{{FindIndividual}.png}


\section{Les contacts moraux}
\label{\detokenize{contacts/legal_entity:les-contacts-moraux}}\label{\detokenize{contacts/legal_entity::doc}}
Un contact moral est une structure ou d’une organisation de personne (entreprise, association, administration, …), à mémoriser.


\subsection{Liste de vos contacts moraux}
\label{\detokenize{contacts/legal_entity:liste-de-vos-contacts-moraux}}
Le menu \sphinxstyleemphasis{Bureautique/Adresses et Contacts/Personnes morales} vous permet de consulter la liste des structures que vous avez déjà enregistrées. Chaque contact moral est associé à une catégorie. Dans cette liste, vous consultez vos structures filtrées par ces catégories.

Depuis cet écran, vous avez aussi la possibilité d’imprimer la liste des structures.

\noindent\sphinxincludegraphics{{ListLegalEntity}.png}


\subsection{Edition d’un contact moral}
\label{\detokenize{contacts/legal_entity:edition-d-un-contact-moral}}
Depuis la liste précédente, vous avez la possibilité de créer une nouvelle structure. Vous pouvez ré\sphinxhyphen{}éditer cette fiche depuis sa visualisation.

\noindent\sphinxincludegraphics{{EditLegalEntity}.png}


\subsection{Visualisation d’un contact moral}
\label{\detokenize{contacts/legal_entity:visualisation-d-un-contact-moral}}
Depuis la liste des personnes morales, vous avez la possibilité de visualiser une structure.

Cela vous permettra de consulter la fiche d’une structure précédemment entrée dans votre base. Vous pouvez modifier cette fiche ou l’imprimer. Si cette personne n’est pas référencée dans d’autre enregistrement de l’application, vous avez la possibilité de la supprimer.

\noindent\sphinxincludegraphics{{ShowLegalEntity}.png}


\subsection{Responsables d’un contact moral}
\label{\detokenize{contacts/legal_entity:responsables-d-un-contact-moral}}
Vous avez la possibilité d’associer une personne physique à votre structure.

Choisissez le nouveau responsable: si la personne n’existe pas dans votre base, vous aurez la possibilité de la créer. Vous pourrez également ajouter une fonction à un responsable défini.

\noindent\sphinxincludegraphics{{ResponsabilityLegalEntity}.png}


\subsection{Recherche d’un contact moral}
\label{\detokenize{contacts/legal_entity:recherche-d-un-contact-moral}}
Le menu \sphinxstyleemphasis{Bureautique/Adresses et Contacts/Recherche de personne morale} vous permet de définir un critère de recherche sur une structure morale.

\noindent\sphinxincludegraphics{{FindLegalEntity}.png}


\section{Configuration et paramétrage}
\label{\detokenize{contacts/configuration:configuration-et-parametrage}}\label{\detokenize{contacts/configuration::doc}}
Dans le menu \sphinxstyleemphasis{Administration/Modules (conf.)} vous avez à votre disposition des outils pour configurer la gestion des contacts.


\subsection{Configuration des contacts}
\label{\detokenize{contacts/configuration:configuration-des-contacts}}
Dans cet écran, vous avez la possibilité de créer ou de modifier une définition de fonction, ou responsabilité, pour associer une personne physique à une structure morale. Vous pouvez créer ou modifier une catégorie de structure morale pour vous aider dans la classification de vos contacts moraux.

Il se peut que vous ayez besoin de préciser des informations supplémentaires pour vos différents contacts. Vous avez ici la possibilité d’ajouter des champs personnels pour chaque type de contacts. Pour ajouter un champ, vous devez simplement donner son titre ainsi que définir son type et éventuellement les compléments nécessaires.
5 types are possibles:
\begin{itemize}
\item {} 
chaîne de texte

\item {} 
nombre entier

\item {} 
nombre à virgule (réel)

\item {} 
valeur Oui/Non (booléen)

\item {} 
choix dans une liste (énumération)

\end{itemize}

Dans le cas de l’énumération, vous devez définir la liste des valeurs possibles (mots) séparées par un point\sphinxhyphen{}virgule.


\subsection{Codes postaux/villes}
\label{\detokenize{contacts/configuration:codes-postaux-villes}}
Cela peux vous aider dans votre saisi de contact, l’outil va automatiquement rechercher la ville (ou les villes) associée(s) avec le code postal que vous entrerez.
Dans cet écran, vous pouvez ajouter des codes postaux manquants.
Par défaut, les codes postaux français et suisses sont insérés.


\chapter{Lucterios courier}
\label{\detokenize{mailing/index:lucterios-courier}}\label{\detokenize{mailing/index::doc}}
Aide relative aux fonctionnalités de courier et publipostage.


\section{Publipostage}
\label{\detokenize{mailing/mailing:publipostage}}\label{\detokenize{mailing/mailing::doc}}
Depuis le menu \sphinxstyleemphasis{Bureatique/Publipostage/Message} vous avez la possibilité de créer un courier de publipostage.


\subsection{Création d’un message}
\label{\detokenize{mailing/mailing:creation-d-un-message}}
Une fois votre message rédigé, vous pouvez lui associé des requetes de destinataires.
C’est requetes de recherches, similaire à celle des outils de recherche de contacts, ne seront évaluées qu’au moment de la génération du courier.
Ainsi, même un contact dernièrement ajouté ou modifié pourra être impacté par ce message.

Il est également possible d’ajouter à votre message un ou plusieurs documents, sauvés dans le \sphinxstyleemphasis{gestionnaire de documentation}.
Ces documents seront transmis en pièces\sphinxhyphen{}jointes dans l’envoie par courriel.

L’option \sphinxstyleemphasis{document(s) ajouté(s) via liens dans le message} permet d’ajouter un ensemble de liens partagés vers vos documents (et non plus des pièces jointes).
Cela permet de gérer des documents de taille importante ou qui risqueraient d’être supprimer par certain gestionnaire de courriel.

\noindent\sphinxincludegraphics{{mailing}.png}


\subsection{Validation \& transmission}
\label{\detokenize{mailing/mailing:validation-transmission}}\begin{description}
\item[{Une fois le message validé vous pouvez:}] \leavevmode\begin{itemize}
\item {} 
Soit généré une sortie PDF de l’ensemble des lettres à envoyer personnalisé avec l’entête de chaque contact

\item {} 
Soit envoyé par courriel si votre configuration est valide. Bien sur, dans ce cas, seul les contacts possédant une adresse seront impacté par cet envoie.

\end{itemize}

\end{description}

De plus, dans le cas d’un envoie par courriel, vous pouvez consulter un rapport de transmission.
Celui\sphinxhyphen{}ci vous indique les courriels envoyés, leur éventuel erreur d’acheminement.

Si votre logiciel est accessible depuis internet, vous pouvez également consulter le nombre de fois que le destinataire à consulter ce message.
Ce mécanisme se base sur l’acceptation, par votre destinataire des images distantes présentent dans le message.

\noindent\sphinxincludegraphics{{transmission}.png}


\section{Configuration du couriel}
\label{\detokenize{mailing/configuration:configuration-du-couriel}}\label{\detokenize{mailing/configuration::doc}}
Vous pouvez configurer ici des réglages pour l’envoi de couriel.

Le serveur SMTP permettra au logiciel d’envoyer directement des messages à vos contacts.
Configurez donc ici les règlages de votre serveur.
Vous pouvez également préciser un \sphinxstyleemphasis{Fichier privé DKIM} et \sphinxstyleemphasis{Sélecteur DKIM} afin de signer vos envoies de courriel.
Les paramètres \sphinxstyleemphasis{durée (en min) d’un lot de courriel} et \sphinxstyleemphasis{nombre de courriels par lot} sont utilisés pour l’envoie des messages en publipostage.

Un bouton \sphinxstyleemphasis{Envoyer} permet de tester vos règlages en envoyant un courriel de test à un destinataire choisi.
Il existe des outils permettant de vérifier si vos messages respectent des règles afin d’éviter d’être considérer comme des “pourriel”.
En autre, l’outil \sphinxurl{https://www.mail-tester.com} (gratuit jusqu’à 3 fois par jour) vous permet, en envoyant un message à l’adresse précisée, de vous établir une note de confiance.

Vous pouvez, entre autre, envoyer d’un nouveau mot de passe de connexion.
N’oubliez pas alors de préciser un petit message d’explication via le paramètres \sphinxstyleemphasis{Message de confirmation de connexion}.


\chapter{Lucterios documents}
\label{\detokenize{documents/index:lucterios-documents}}\label{\detokenize{documents/index::doc}}
Aide relative aux fonctionnalités de gestion documentaire.


\section{Fichiers partagés}
\label{\detokenize{documents/shared_document:fichiers-partages}}\label{\detokenize{documents/shared_document::doc}}

\subsection{Liste des documents}
\label{\detokenize{documents/shared_document:liste-des-documents}}
Le menu \sphinxstyleemphasis{Bureautique/Gestion documentaire/Documents} vous permet de consulter la liste des fichiers que vous avez déjà enregistrés. Pour vous aider à retrouver vos documents, la liste est classifiée par un ensemble de dossiers et de sous\sphinxhyphen{}dossiers et une description vous donne un petit résumé.

Vous avez aussi la possibilité d’ajouter un sous\sphinxhyphen{}dossier ou de modifier les propriétés du dossier courant.

\noindent\sphinxincludegraphics{{listdoc}.png}

Suivant vos permissions, vous pouvez extraire votre fichier pour le consulter, le modifier et éventuellement ré\sphinxhyphen{}injecter vos corrections.

De plus, l’outil mémorisera l’utilisateur et la date de création du document ainsi que les informations relatives à la dernière modification.

\noindent\sphinxincludegraphics{{showdoc}.png}

Depuis la fiche du document, il vous est possible d’activer un lien de téléchargement.
Ce lien web peut être transmis à une personne tiers, n’ayant aucun droit d’accès à votre logiciel, afin de télécharger le document.
\sphinxstylestrong{Attention:} Votre instance doit être accessible sur internet pour que le lien puisse fonctionner depuis n’importe où.


\subsection{Recherche de documents}
\label{\detokenize{documents/shared_document:recherche-de-documents}}
Le menu \sphinxstyleemphasis{Bureautique/Gestion documentaire/Recherche de document} vous permet de définir un critère de recherche sur un document.

Une fois validé, l’outil va rechercher dans la base toutes les fichiers correspondants à ces critères.


\section{Editeur de documents}
\label{\detokenize{documents/editor:editeur-de-documents}}\label{\detokenize{documents/editor::doc}}
Il est possible de configurer l’outil afin de pouvoir éditer certain document directement via l’interface « en ligne ».

Des outils d’édition, libres et gratuits, sont actuellement configurable afin de les utiliser pour consulter et modifier des documents.
\_Note:\_ Ces outils sont gérés par des équipes complètement différentes, il se peux que certain de leur comportement ne correspondent pas à vos attentes.


\subsection{etherpad}
\label{\detokenize{documents/editor:etherpad}}
Editeur pour document textuel.
\begin{description}
\item[{Site Web}] \leavevmode
\sphinxurl{https://etherpad.org/}

\item[{Installation}] \leavevmode
Le tutoriel de framasoft explique bien comment l’installer
\sphinxurl{https://framacloud.org/fr/cultiver-son-jardin/etherpad.html}

\item[{Configurer}] \leavevmode
Editer le fichier « settings.py » contenu dans le répertoire de votre instance.
Ajouter et adapter la ligne ci\sphinxhyphen{}dessous:
\begin{itemize}
\item {} 
url : adresse d’accès d’etherpad

\item {} 
apikey : contenu de la clef de sécurité (fichier APIKEY.txt contenu dans l’installation d’etherpad)

\end{itemize}

\end{description}

\begin{sphinxVerbatim}[commandchars=\\\{\}]
\PYG{c+c1}{\PYGZsh{} extra}
\PYG{n}{ETHERPAD} \PYG{o}{=} \PYG{p}{\PYGZob{}}\PYG{l+s+s1}{\PYGZsq{}}\PYG{l+s+s1}{url}\PYG{l+s+s1}{\PYGZsq{}}\PYG{p}{:} \PYG{l+s+s1}{\PYGZsq{}}\PYG{l+s+s1}{http://localhost:9001}\PYG{l+s+s1}{\PYGZsq{}}\PYG{p}{,} \PYG{l+s+s1}{\PYGZsq{}}\PYG{l+s+s1}{apikey}\PYG{l+s+s1}{\PYGZsq{}}\PYG{p}{:} \PYG{l+s+s1}{\PYGZsq{}}\PYG{l+s+s1}{jfks5dsdS65lfGHsdSDQ4fsdDG4lklsdq6Gfs4Gsdfos8fs}\PYG{l+s+s1}{\PYGZsq{}}\PYG{p}{\PYGZcb{}}
\end{sphinxVerbatim}
\begin{description}
\item[{Usage}] \leavevmode\begin{description}
\item[{Dans le gestionnaire de document, vous avez plusieurs action qui apparait alors}] \leavevmode\begin{itemize}
\item {} 
Un bouton « + Fichier » vous permettant de créer un document txt ou html

\item {} 
Un bouton « Editeur » pour ouvrir l’éditeur etherpad.

\end{itemize}

\end{description}

\end{description}

\noindent\sphinxincludegraphics{{etherpad}.png}


\subsection{ethercalc}
\label{\detokenize{documents/editor:ethercalc}}
Editeur pour tableau de calcul.
\begin{description}
\item[{Site Web}] \leavevmode
\sphinxurl{https://ethercalc.net/}

\item[{Installation}] \leavevmode
Sur le site de l’éditeur, une petit explication indique comment l’installer.

\item[{Configurer}] \leavevmode
Editer le fichier « settings.py » contenu dans le répertoire de votre instance.
Ajouter et adapter la ligne ci\sphinxhyphen{}dessous:
\begin{itemize}
\item {} 
url : adresse d’accès d’ethercal

\end{itemize}

\end{description}

\begin{sphinxVerbatim}[commandchars=\\\{\}]
\PYG{c+c1}{\PYGZsh{} extra}
\PYG{n}{ETHERCALC} \PYG{o}{=} \PYG{p}{\PYGZob{}}\PYG{l+s+s1}{\PYGZsq{}}\PYG{l+s+s1}{url}\PYG{l+s+s1}{\PYGZsq{}}\PYG{p}{:} \PYG{l+s+s1}{\PYGZsq{}}\PYG{l+s+s1}{http://localhost:8000}\PYG{l+s+s1}{\PYGZsq{}}\PYG{p}{\PYGZcb{}}
\end{sphinxVerbatim}
\begin{description}
\item[{Usage}] \leavevmode\begin{description}
\item[{Dans le gestionnaire de document, vous avez plusieurs action qui apparait alors}] \leavevmode\begin{itemize}
\item {} 
Un bouton « + Fichier » vous permettant de créer un document csv, ods ou xmlx

\item {} 
Un bouton « Editeur » pour ouvrir l’éditeur ethercalc.

\end{itemize}

\end{description}

\end{description}

\noindent\sphinxincludegraphics{{ethercalc}.png}


\section{Configuration}
\label{\detokenize{documents/configuration:configuration}}\label{\detokenize{documents/configuration::doc}}
Dans le menu \sphinxstyleemphasis{Administration/Module (conf)/Dossiers} vous avez à votre disposition un ensemble d’outils pour configurer la gestion documentaire.


\subsection{Dossiers}
\label{\detokenize{documents/configuration:dossiers}}
Dans cet écran vous avez la possibilité de créer ou de modifier des dossiers de classement documentaire.

\noindent\sphinxincludegraphics{{configuration}.png}

En associant judicieusement un dossier comme sous\sphinxhyphen{}dossier d’un parent, vous pouvez vous définir une arborescence de classement.

Vous associez à chaque dossier un ensemble de groupe de droits pour la visualisation et la modification des fichiers. Seuls les utilisateurs appartenant aux groupes de visualisation pourront consulter les documents de cette catégorie. Seuls les utilisateurs appartenant aux groupes de modification pourront corriger les documents de cette catégorie.


\chapter{Coeur Lucterios}
\label{\detokenize{CORE/index:coeur-lucterios}}\label{\detokenize{CORE/index::doc}}
Aide relative aux fonctionnalités générales de cet outil de gestion.


\section{Mot de passe}
\label{\detokenize{CORE/password:mot-de-passe}}\label{\detokenize{CORE/password::doc}}
Le menu \sphinxtitleref{Général/Mot de passe} vous permet de changer le mot de passe d’accès de l’utilisateur courant.

\noindent\sphinxincludegraphics{{password}.png}

Pour plus de sécurité, nous vous conseillons d’utiliser un mot de passe comprenant des lettres et des chiffres et ne constituant pas un mot compréhensible.


\section{Les groupes}
\label{\detokenize{CORE/groups:les-groupes}}\label{\detokenize{CORE/groups::doc}}
Le menu \sphinxtitleref{Administration/Gestion des Droits/Les groupes} vous permet de créer, modifier ou supprimer un groupe de droits.

\noindent\sphinxincludegraphics{{group}.png}

Un groupe de droits réunit un ensemble d’autorisations aux actions de l’application.

\noindent\sphinxincludegraphics{{group_modify}.png}


\section{Les utilisateurs}
\label{\detokenize{CORE/users:les-utilisateurs}}\label{\detokenize{CORE/users::doc}}
Le menu \sphinxtitleref{Administration/Gestion des Droits/Les utilisateurs} vous
permet de créer, modifier ou désactiver un utilisateur de l’application. Un
utilisateur définit un droit de connexion au logiciel.

\noindent\sphinxincludegraphics{{users}.png}

Depuis cette liste, vous pouvez créer ou modifier l’utilisateur: son
alias, son nom et son mot de passe. A cela, vous lui ajouter des groupes et
des permissions suplémentaires éventuelles afin de définir son niveau
d’accès au logiciel. Vous pouvez aussi désactiver un utilisateur pour lui
interdire l’accès à l’application.

\noindent\sphinxincludegraphics{{user_info}.png}

\noindent\sphinxincludegraphics{{user_permissions}.png}


\section{L’architecture du logiciel}
\label{\detokenize{CORE/architecture:l-architecture-du-logiciel}}\label{\detokenize{CORE/architecture::doc}}
Depuis le commencement de ce logiciel, les développeurs ont voulu que cette application puisse avoir une architecture ouverte permettant des évolutions les plus larges.



\renewcommand{\indexname}{Index}
\printindex
\end{document}